%!TEX root = ../main.tex

\thispagestyle{empty}
\vspace*{4cm}
\begin{Huge}
    \bfseries \selectfont \hfill \scshape Abstract\\\\
    \rule[0.5ex]{\linewidth}{1pt}
\end{Huge}

The prime number theorem is the assetion that in the limit, the quotient $\dfrac{\pi(x)\log x}{x}$ goes to 1, which means that $\pi(x)\thicksim \dfrac{x}{\log x}$, where $\pi(x)$ is the prime counting function. In arithmetic progressions $a+kq$ with $(a,q)=1$, we have that $\pi(a,q,x)$; the prime counting function restricted to the progression, has the asymptotic behavior $\pi(a,q,x)\thicksim \dfrac{x}{\phi(q)\log x}$, meaning that primes are uniformly distributed among the residue classes modulo $q$. In this work, we will present the proof of this result, the underlying ideas, and applications. For this, we will make use of Tauberian theory, which will allow us to present a detailed and concise proof, followed by studying the non-vanishing of $L(\chi,s)$ and some properties of Dirichlet characters and series.\\\\\\

\begin{Huge}
    \bfseries \selectfont \scshape Resumen\\\\
    \rule[0.5ex]{\linewidth}{1pt}
\end{Huge}

El teorema de los números primos nos dice que, en el límite, el cociente $\dfrac{\pi(x)\log x}{x}$ tiende a 1, es decir, que $\pi(x)\thicksim \dfrac{x}{\log x}$, donde $\pi(x)$ es la función contadora de primos. En progresiones aritméticas $a+kq$, con $(a,q)=1$, tenemos que $\pi(a,q,x)$, la función contadora restringida a la progresión, tiene el comportamiento asintótico $\pi(a,q,x)\thicksim \dfrac{x}{\phi(q)\log x}$, es decir, los primos se distribuyen uniformemente en las clases de residuos módulo $q$. En este trabajo se presentará la prueba de este resultado, las ideas subyacentes y aplicaciones. Para esto, haremos uso de la teoría Tauberiana, lo que nos permitirá presentar una prueba detallada y corta, que se basará en la no nulidad de $L(\chi,s)$ y en algunas propiedades de los caracteres y series de Dirichlet.

