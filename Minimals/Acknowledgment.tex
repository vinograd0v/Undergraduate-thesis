%!TEX root = ../main.tex

\newpage
\thispagestyle{empty}
\begin{Huge}
    \bfseries \selectfont \scshape \hfill Agradecimientos \\\\
    \rule[0.5ex]{\linewidth}{1pt}
    \vspace*{1cm}
\end{Huge}

Por muchos años solía creer imposible que algún día me graduaría de matemático. Al momento de entregar esta tesis vienen a mí muchos recuerdos del largo y duro proceso que fue para un muchacho de una región, aparentemente olvidada para el país, llegar a Bogotá y adaptarse a las exigencias académicas de esta universidad. Sin lugar a dudas hay algo  de cierto en lo que pensaba, no podría haber llegado aquí sin el apoyo  de muchas personas en mi vida, esta tesis no hubiera sido posible si estas personas no me hubieran apoyado de la forma en que lo hicieron.\\

Así pues, quisiera comenzar agradeciendo a mi familia, primero a mis padres (Marco y Raiza) por su apoyo incondicional y por creer en mí durante estos largos años, papá, sin ti nada de esto hubiera sido posible.\\

También quisiera agradecer a mi hermano James por haber sido otro padre para mí y a mi hermano Juan, quien siempre fue mi ejemplo, mi compañía y mi mejor amigo, fue gracias a él que conocí la educación superior. Quisiera agradecer a mis hermanos (Luis y Fabio) por su amistad y su apoyo a lo largo de los años, a Jessica y Kamila por sacarme de la casa cuando probablemente la matemática ya me tenía el cerebro estallado, también quisiera agradecerle a mi tía Marleny por haber sido un gran apoyo para mi padre y para mí. Gracias a todos, nunca han dejado de estar a mi lado, sus consejos y su afecto los atesoro en mi corazón.\\
Quisiera agradecer a Karolina por su apoyo, amor y comprensión a lo largo de los años, por haber  creído en mí cuando yo no lo hice. Sin lugar a dudas, sin ella este trabajo tampoco hubiera sido posible.\\

También quiero agradecer a mis amigos, a los viejos (Andrés, Iván y Viuche) con quienes he compartido casi toda la carrera y a los nuevos (Alejandra, Santiago, Sergio y Sandra). Su amistad y su apoyo a lo largo de los años ha sido fundamental, gracias por todas las charlas sobre matemáticas y por divagar conmigo sobre cualquier cosa, por las risas, por todas las veces que salimos luego de clases y por la cantidad no contable de noches que trasnochamos estudiando, gracias a ustedes soy mucho mejor matemático de lo que pensé que podría ser y la carrera no ha sido todo un infierno. Me honra haber conocido tan buenos matemáticos y haber hecho tan buenos amigos a lo largo de los años.\\
Quisiera agradecerles particularmente a Santiago y Alejandra por estos últimos años en los que fueron indispensables, por haber sido mis compañeros de trabajo en la parte más difícil de la carrera y por la amistad que hemos construido.\\

Adicionalmente quisiera expresar mi gratitud a Sergio y Santiago por su constante ayuda a revisar este largo trabajo, particularmente a Santiago, aunque no se  exprese en forma puntual en las páginas de este texto, muchas de sus ideas fueron de gran ayuda cuando me encontraba estancado en una demostración.\\

<<<<<<< HEAD
Finalmente quiero agradecer a los profesores del departamento de matemáticas, particularmente al profesor John Jaime Rodriguez, por haber sido un director paciente, comprensivo y por todo lo que aprendí de él a largo del tiempo, sin lugar a duda llevó a que esta tesis sea lo que es.\\
=======
También quiero darle gracias a  mis amigos de física (Lucho y Juanpa) por las veces salimos juntos, en las que hablamos de física, matemáticas, de la vida, por todo lo que me apoyaron y los buenos momentos que hemos vivido juntos.\\

Quiero agradecer a los profesores del departamento de matemáticas, particularmente al profesor John Jaime Rodriguez, por ser un director paciente, comprensivo y por todo lo que aprendí de él a largo del tiempo, sin lugar a duda llevó a que esta tesis sea lo que es.\\
>>>>>>> 9fa4a234e0f0551414f9cf0b7e43e26f777b6db9

Gracias a todos.
\thispagestyle{empty}
