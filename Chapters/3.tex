%!TEX root = ../main.tex

\thispagestyle{empty}
\vspace{-0.7cm}

\cleanchapterquote{...Dirichlet creó una parte nueva en las matemáticas, la aplicación de las series infinitas que Fourier ha introducido en la teoría del calor en la exploración de las propiedades de los números primos. Él ha descubierto una variedad de teoremas que ... son los pilares de las nuevas teorías}{C. G. J. Jacobi}{}

El teorema de Dirichlet afirma que dados $a,n\in  \N$ tal que $(a,n)=1$, hay infinitos primos de la forma $a, a+n, a+2n, a+3n,\ldots$, el primer resultado sobre la infinitud de los números primos se remonta a Euclides. Supongamos que hay una cantidad  finita de primos, podemos contarlos... $p_1,p_2,\ldots,p_n$, note que $p_1p_2\ldots p_n+1$ es primo ya que si $p_i\mid p_1p_2\ldots p_n+ 1$ para algún $1\leq i\leq n$, entonces

$$1=p_i(K-(p_1\ldots p_{i-1}p_{i+1}\ldots p_n)),$$

luego $p_i\mid 1$, una contradicción, es decir, siempre podemos construir un primo $p_{n+1}$ con los $n$ primos anteriores, entonces son infinitos.\\

Intentemos replicar este argumento para probar que hay infinitos primos de la forma $4k+1$, supongamos que hay finitos primos de la forma $4k+1$, digamos $p_1,p_2,\ldots,p_n$, debemos construir un nuevo primo de  la forma $4k+1$ para que funcione el argumento de Euclides, sin embargo la expresión  $p_1p_2\ldots p_n+1$ no siempre es de la forma $4k+1$, por ejemplo $5\times13+1=66$, que es congruente a 2 módulo 4, de hecho con esta expresión siempre conseguimos pares. Requerimos una expresión nueva, por ejemplo, podríamos hacer $2p_1\ldots p_n+1$, pero también falla, note que $2\times5\times13+1=131$ que es un primo de la forma $4k+3$.\\

\begin{prop}

Sea $n\in \Z$, todo divisor primo impar de $n^2+1$ es de la forma $4k+1$. 
\end{prop}

\begin{proof}
Suponga que existe $p=4k+3$ primo tal que $p\mid n^2+1$, entonces $n^2\equiv -1 \pmod{p}$, luego por el pequeño teorema de Fermat

$$\pmod{p}: 1\equiv n^{ p-1}\equiv (n^{2})^{2k+1}\equiv (-1)^{2k+1}\equiv -1,$$

así $2\equiv 0 \pmod{p}$, esto es $p=2$, contradicción.

\end{proof}

Con el teorema anterior sabemos que la expresión que buscamos es $N=(2p_1\ldots p_n)^2+1$ ya que de esta forma obtenemos un número cuyos divisores primos son de la forma $4k+1$, basta ver que ningún $p_i$ con $1\leq i\leq n$ divide a $N$.\\

Supongamos que $p_i\mid N$, luego $p_i(K-4p_1^2\ldots p_i\ldots p_n^2)=1$ una contradicción, entonces $N$ es un primo de la forma $4k+1$. Continuar replicando este argumento es inviable cuando trabajamos con primos módulo un entero $n$ arbitrario, además no nos sirve para atacar el panorama general, la prueba de este teorema llegaría de una idea totalmente distinta...


\begin{theorem}[Euler]
La serie $\displaystyle\sum_p \dfrac{1}{p}$ diverge.
\end{theorem}

\begin{proof}
Por el producto de Euler:
    $$
\log (\zeta(s))=\sum_p^{\infty}\left(\displaystyle\sum_{k=1}^{\infty} \dfrac{1}{k(p)^{k s}}\right)=\sum_p
\dfrac{1}{p^s}+\sum_{p}\left(\sum_{k=2}^{\infty}\dfrac{1}{kp^{ks}}\right), \quad \Re(s)>1$$

note que:

\begin{align*}
    \sum_{p}\left(\sum_{k=2}^{\infty}\dfrac{1}{kp^{ks}}\right)&\leq \sum_{p}\left(\sum_{k=2}^{\infty}\dfrac{1}{p^{ks}}\right)\\
    &\leq\sum_{p}\dfrac{1}{p^s}\left(\dfrac{1}{p^s-1}\right)\\
    &\leq \sum_{p}\dfrac{1}{p^s(p^s-1)}\leq \sum_{n=1}^{\infty}\dfrac{1}{n^s(n^s-1)}.
\end{align*}

\footnote{La convergencia de esta serie nos llegó haciendo cuentas en una clase de estructuras algebraicas, Santiago me dijo ``Mateo esa serie es geométrica''.}La última serie converge siempre que $\Re(s)>\dfrac{1}{2}$, entonces por la divergencia de la serie armónica, tomando el límite cuando $s\to 1^+$

$$
\infty=\lim_{s\to 1^+}\log(\zeta(s))=\lim_{s\to 1^+}\sum_p
\dfrac{1}{p^s}+\sum_{p}\left(\sum_{k=2}^{\infty}\dfrac{1}{kp^{ks}}\right)=\sum_p
\dfrac{1}{p}+ O(1),$$

lo que nos dice que la suma de los recíprocos de los primos diverge y por lo tanto podemos decir que los primos son infinitos. 
\end{proof}

La idea de Dirichlet es replicar este argumento de Euler para probar que hay infinitos primos de la forma $4k+1$, para ello define la siguiente serie

$$
L(s, \chi)=\sum_{n=1}^{\infty} \frac{\chi(n)}{n^s},
$$

donde, $\chi(n)$ es la función indicadora:

$$
\chi(a)= \begin{cases}0 & \text { si } a \text { es par } \\ 1 & \text { si } a \equiv 1 \bmod 4 \\ -1 & \text { si } a \equiv 3 \bmod 4\end{cases}
$$

note que $\chi$ es completamente multiplicativa, entonces

$$
L(s, \chi)=\prod_p \frac{1}{1-\chi(p) p^{-s}}.
$$

Siguiendo los pasos de la prueba anterior obtenemos

\begin{align*}
    \log L(s, \chi)&=\sum_p \frac{\chi(p)}{p^s}+\sum_p\sum_{k=2}^{\infty}\dfrac{\chi(k)}{kp^{ks}}\\
    &=\sum_{p\equiv 1(4)}\frac{1}{p^s}- \sum_{p\equiv 3(4)}\frac{1}{p^s}+g_1(s,\chi)
,\end{align*}

donde no es difícil ver que $g_1(s,\chi)$ es convergente cuando $s\to 1^{+}$ ya que $|\chi(n)|\leq 1$, además

$$\log \zeta(s)=\sum_p \frac{1}{p^s}+g(s),$$

así

$$\begin{aligned}
& \log \zeta(s)+\log L(s, \chi)=2 \sum_{p \equiv 1(4)} \frac{1}{p^s}+\left(\frac{1}{2^s}+g(s)+g_1(s, \chi)\right), \\
& \log \zeta(s)-\log L(s, \chi)=2 \sum_{p \equiv 3(4)} \frac{1}{p^s}+\left(\frac{1}{2^s}+g(s)-g_1(s, \chi)\right).
\end{aligned}$$

Tomando $\lim_{s\to 1^{+}}$ como antes se verifica que hay infinitos primos de la forma $4k+1$ y $4k+3$ ya que el término restante converge, sin embargo debemos garantizar algo, que $L(1,\chi)$ converge y es distinto de 0. Para esto note  que

\begin{align*}
    \sum_{n=1}^{\infty}\dfrac{\chi(n)}{n}=\sum_{n=1}^{\infty}\dfrac{(-1)^n}{2n+1}=\arctan(1)=\dfrac{\pi}{4}
.\end{align*}

Este camino parece fructífero, intentemos replicar esto en un caso general, sea $f(n)$ la función característica de la progresión aritmética, es decir

$$
f(n)=\left\{\begin{array}{lll}
1, & n \equiv a & \pmod{m} \\
0, & n \not \equiv a & \pmod{m}
\end{array}\right.
$$

en el caso de que $f(n)$ sea completamente multiplicativa tendríamos un producto de Euler

$$
\sum_{n=1}^{\infty} \frac{f(n)}{n^s}=\prod_p\left(1-\frac{f(p)}{p^s}\right)^{-1}, \quad \Re(s)>1
$$

y así por argumentos análogos a los de Euler se tendría que

$$\log \left(\sum_{n=1}^{\infty} \frac{f(n)}{n^s}\right)=\sum_{p \equiv a(m)} \frac{1}{p^s}+O(1)$$

Lamentablemente, $f(n)$ generalmente no es multiplicativa\\

Para resolver este inconveniente estudiaremos los caracteres de un grupo y en particular los caracteres de Dirichlet, estos veremos que poseen propiedades de ortogonalidad, lo que nos permitirá hacer ¡Análisis de Fourier! y representar a la función $f$ característica de la progresión en su serie de Fourier como una combinación lineal finita de funciones completamente multiplicativas (los caracteres), con esto en mente veamos primero unos preliminares sobre caracteres que necesitaremos en la prueba.

\section{Caracteres y el teorema de Dirichlet}

Primero vamos a presentar definición formal de la idea de carácter.\\

\begin{definition}
Sea $G$ un grupo, $\chi$ es un carácter de $G$ si $\chi: G\to \C^{\times}$ y satisface que para todo $a,b\in G$, $\chi(ab)=\chi(a)\chi(b)$, es decir, un homomorfismo de $G$ en $\C^{\times}$.
\end{definition}

 El homomorfismo trivial que mapea a todo $g\in G$ al 1 lo llamaremos ``carácter trivial'' denotado $\chi_0$.\\

 \begin{definition}
 Dado un grupo $G$, definimos el conjunto de todos los caracteres de $G$, denotado como $\widehat{G}$. También definimos la multiplicación en $\widehat{G}$ como

$$\chi_1\chi_2(g)=\chi_1(g)\chi_2(g) \quad \text{para todo } \chi_1,\chi_2\in \widehat{G} \quad \text{y } g\in G$$

 \end{definition}

Decimos $e(x)$ para denotar $e^{2 \pi i x}$. Por ejemplo, $e(1 / n)$ es una raíz $n$-ésima de la unidad. Una caracterización de los caracteres de un grupo cíclico finito es la siguiente.

\begin{theorem}
Sea $G$ un grupo cíclico de orden $n$ generado por $g$, $G=\langle g\rangle$, entonces hay exactamente $n$ caracteres $\chi_0,\ldots,\chi_{n-1}$ de $G$, dados por $\chi_m(g^k)=e\left((mk)/n\right)$ para todo $0\leq m\leq n-1$ y $k\in \Z$, esto es 
$$\chi_0(g^k)=1, \chi_1(g^k)=e(k/n),\chi_2(g^k)=e(2k/n),\ldots,\chi_{n-1}(g^k)=e((n-1)k/n).$$
\end{theorem}

\begin{proof}
Sea $\chi$ un carácter de $G$, tenemos que $\chi(g)^n=\chi(g^n)=\chi(1)=1$, esto es que $\chi(g)$ es una raíz $n-$ésima de la unidad, luego $\chi(g)=e(m/n)$ para algún $0\leq m\leq n-1$, dado  que $G$ es cíclico y generado por $g$, entonces $\chi$ está totalmente determinado por $\chi(g)$ y $\chi(g)^k=\chi(g^k)=e((km)/n)$ para todo $k\in \Z$, esto es $\chi=\chi_m$, Así si $\chi$ es un carácter de $G$, $\chi$ debe ser uno de los $\chi_0,\ldots,\chi_{n-1}$.Ahora veamos que $\chi_0,\ldots,\chi_{n-1}$ están bien definidos y son caracteres distintos de $G$.\\

En efecto si $g^{k_1}=g^{k_2}$ entonces $k_1\equiv k_2\pmod{n}$, luego $\chi_m(g^{k_1})=e\left((mk_1)/n\right)=e\left((mk_2)/n\right)=\chi_m(g^{k_2})$ para todo $0\leq m\leq n-1$, por tanto $\chi_0,\ldots,\chi_{n-1}$ están bien definidos en $G$. Dados $0\leq m_1,m_2\leq n-1$ con $m_1\neq m_2$, entonces
$$\displaystyle\chi_{m_1}(g)=e\left(\frac{m_1}{n}\right)\neq e\left(\frac{m_2}{n}\right)=\chi_{m_2}(g),$$

como $\chi_{m_1}$ y $\chi_{m_2}$ están totalmente determinados por $\chi_{m_1}(g)$ y $\chi_{m_2}(g)$, entonces $\chi_0,\ldots\chi_{n-1}$ son todos distintos.\\

Finalmente debemos ver que dado $\chi_m$ con $0\leq m\leq n-1$, $\chi_m$ es homomorfismo de $G$ en $\C^{\times}$, en efecto dados $a,b\in G$, $a=g^{k_1}$ y $b=g^{k_2}$. Note que

\begin{align*}
\chi_m(ab)&=\chi_m(g^{k_1}g^{k_2})\\
&=\chi_m(g^{k_1+g_2})\\
&=e\left(\frac{m(k_1+k_2)}{n}\right)\\
&=e\left(\frac{mk_1}{n}\right)e\left(\frac{mk_2}{n}\right)\\
&=\chi_m(a)\chi_m(b)
.\end{align*}

Esto es, $\chi_m$ es homomorfismo.
\end{proof}

\begin{theorem}
Sea $G$ un grupo, el  conjunto $\widehat{G}$ es un grupo abeliano bajo la multiplicación.
\end{theorem}

\begin{proof}
El carácter principal $\chi_0$ de $G$ es la identidad de $\widehat{G}$ ya que dado $a\in g$ $\chi_m\chi_0(a)=\chi_m(a)\chi_0(a)=\chi_m(a)$ con $0\leq m\leq n-1$, además note que la función $\chi^{-1}=1/\chi(g)$ es también un carácter y por lo tanto tenemos inversos en $\widehat{G}$, Además $\widehat{G}$ es cerrado bajo la multiplicación, note que:

\begin{align*}
     \chi_{m_1}\chi_{m_2}(ab)&=\chi_{m_1}(ab)\chi_{m_2}(ab)\\
     &=\chi_{m_1}(a)\chi_{m_1}(b)\chi_{m_2}(a)\chi_{m_2}(b)\\
     &=\chi_{m_1}\chi_{m_2}(a)\chi_{m_1}\chi_{m_2}(b)
      .\end{align*}

    La conmutatividad y asociatividad se siguen de manera análoga usando la conmutatividad y asociatividad de $\C^{\times}$.\\
\end{proof}

\begin{note}
Por el teorema anterior, dado que $\widehat{G}$ es grupo, lo llamaremos grupo dual de $G$.\\


La inversa de $\chi$ en la prueba del anterior, denotada  como $\chi^{-1}$ en algunos casos se escribe también como $\overline{\chi}$, la razón de esto es que si $G$ es un grupo abeliano finito entonces $|\chi(g)|=1$, así $\chi^{-1}(g)=1/\chi(g)=\overline{\chi(g)}$ por propiedad de la norma en los complejos. Así la función $\overline{\chi}$ definida como $\overline{\chi}(g)=\overline{\chi(g)}$ y la función $\chi^{-1}$ son iguales. \cite{pongsriiam2023analytic}
\end{note}

\begin{theorem}
Sea $G=\left\langle g\right\rangle$ un grupo cíclico de orden $n$, entonces:

\begin{itemize}
\item[(i)] Dado un carácter $\chi$ de $G$

$$
\sum_{g \in G} \chi(g)= \begin{cases}n, & \text { si } \chi=\chi_0 \\ 0, & \text { e.o.c }\end{cases}
$$

\item[(ii)] Dado $a \in G$

$$
\sum_{\chi \in \widehat{G}} \chi(a)= \begin{cases}n, & \text { si } a=1 \\ 0, & \text { e.o.c }\end{cases}
$$

\item[(iii)] $\widehat{G}$ es un grupo cíclico generado por $\chi_1$.
\end{itemize}

\end{theorem}


\begin{proof}
(i) Note que dado un carácter $\chi$ no trivial de $G$, existe un $0\leq m\leq n-1$ tal que $\chi=\chi_m$. Si $m\neq 0$

$$\sum_{g \in G} \chi(g)=\sum_{k=0}^{n-1} \chi_m\left(g_0^k\right)=\sum_{k=0}^{n-1} e\left(\frac{m k}{n}\right)=\sum_{k=0}^{n-1}\left(e\left(\frac{m}{n}\right)\right)^k=\dfrac{1-e(m)}{1-e\left(\dfrac{m}{n}\right)},$$

y $1-e(m)=1-cos(2\pi m)=0$ para todo $m=1,\ldots,n-1$, si $m=0$ entonces

$$\sum_{k=0}^{n-1} \chi_m\left(g_0^k\right)=\sum_{k=0}^{n-1} 1=n.$$

(ii) Sea $a\in G$, $a=g^{k}$ para algún $k=0,1,\ldots,n-1$, en efecto

$$\sum_{\chi \in \widehat{G}} \chi(a)=\sum_{m=0}^{n-1} \chi_m(a)=\sum_{m=0}^{n-1} \chi_m\left(g^k\right)=\sum_{m=0}^{n-1} e\left(\frac{m k}{n}\right),$$

análogamente se ve que esta suma es $n$ si $k=0$ y es 0 si $k\neq 0$, y el valor $k=0$ corresponde a $g^{0}=1$. Para (iii) note que por el teorema 2.3

$$\widehat{G}=\left\{\chi_0, \chi_1, \ldots, \chi_{n-1}\right\}=\left\{\chi_1^0, \chi_1, \chi_1^2, \ldots, \chi_1^{n-1}\right\}=\left\langle\chi_1\right\rangle.$$

Esto completa la prueba.

\end{proof}

Observe que el teorema anterior nos permite establecer lo siguiente, dado $G$ un grupo cíclico de orden $n$, entonces $G$ es isomorfo a su grupo dual $\widehat{G}$, esto no es exclusivo de los grupos cíclicos, sabemos que todo grupo abeliano finito se puede escribir como suma directa de grupos cíclicos, Dirichlet probó lo siguiente

\begin{theorem}[Dualidad]\label{dualidad}
Sea $G$ un grupo abeliano finito de orden $n$, entonces $G$ tiene exactamente $n$ caracteres. Más aún, sea

$$
G=\mathbb{Z} / m_1 \mathbb{Z} \oplus \mathbb{Z} / m_2 \mathbb{Z} \oplus \cdots \oplus \mathbb{Z} / m_{\ell} \mathbb{Z}
,$$

entonces los caracteres de $G$ son de la forma

$$
\chi\left(\left(a_1, a_2, \ldots, a_{\ell}\right)\right)=e\left(\frac{a_1 k_1}{m_1}\right) e\left(\frac{a_2 k_2}{m_2}\right) \cdots e\left(\frac{a_{\ell} k_{\ell}}{m_{\ell}}\right),
$$

donde $k_i=0,1, \ldots, m_i-1$ para todo $i=1,2, \ldots, \ell$. Adicionalmente, $\widehat{G}$ es un grupo abeliano finito de orden $n,|\widehat{G}|=|G|=m_1 m_2 \cdots m_k$, y $\widehat{G}$ es isomorfo a $G$:

\begin{equation}
\begin{aligned}
\widehat{G} & \simeq \widehat{\mathbb{Z} / m_1 \mathbb{Z}} \times \widehat{\mathbb{Z} / m_2 \mathbb{Z}} \times \cdots \times \widehat{\mathbb{Z} / m_{\ell} \mathbb{Z}} \\
& \simeq \mathbb{Z} / m_1 \mathbb{Z} \oplus \cdots \oplus \mathbb{Z} / m_{\ell} \mathbb{Z} \simeq G.
\end{aligned}
\end{equation}

\end{theorem}

\begin{proof}
Primero consideremos el caso en que $G$ es la suma directa de $\mathbb{Z} / m_i \mathbb{Z}$ para cada $i=1,2,3, \ldots, \ell$, sea $e_i=(0,0, \ldots, 1,0,0, \ldots, 0)$ donde 1 está en la posición $i$. Entonces $e_i$ tiene orden $m_i$ y  por lo tanto $\chi\left(e_i\right)$ es una $m_i$-ésima raíz de la unidad, así pues

\begin{align*}
    \chi\left(\left(a_1,\ldots,a_{\ell}\right)\right)&=\chi(a_1e_1+\ldots +a_{\ell}e_{\ell})\\
    &=\chi(e_1)^{a_1}\ldots\chi(e_2)^{a_2}\ldots\chi(e_{\ell})^{a_{\ell}}\\
    &=e\left(\frac{a_1k_1}{m_1}\right)e\left(\frac{a_2k_2}{m_2}\right)\ldots e\left(\frac{a_{\ell}k_{\ell}}{m_{\ell}}\right)
,\end{align*}

donde $k_i=0,1, \ldots, m_i-1$ para todo $i=1,2, \ldots, \ell$. De esto además vemos que $\chi$ se escribe como un producto de caracteres de $\mathbb{Z} / m_1 \mathbb{Z}, \ldots, \mathbb{Z} / m_{}\mathbb{Z}$, esto es, se tiene (2.1) y $|\widehat{G}|=$ $|G|=m_1 m_2 \cdots m_k$ dado que hay $m_i$ posibles elecciones para $k_i$. Como $G$ es abeliano finito entonces:

$$
G \simeq \mathbb{Z} / m_1 \mathbb{Z} \oplus \cdots \oplus \mathbb{Z} / m_{\ell}\mathbb{Z}.
$$

lo que completa la prueba.
\end{proof}


\subsection{Ortogonalidad de los caracteres}

Los caracteres de un grupo abeliano finito satisfacen relaciones de ortogonalidad, cuando $G=(\Z/m\Z)^{\times}$ estas relaciones de ortogonalidad nos permiten hacer análisis de Fourier, tenemos transformada y representación en ``serie de Fourier'', en nuestro caso esta serie será en realidad una suma finita, por lo que no nos preocuparemos por  la convergencia de la misma. Las relaciones de ortogonalidad que veremos en esta sección son generalizaciones del teorema 2.5, recordemos que la idea es poder expresar la función característica de una clase de residuos módulo $m$ como combinación lineal de caracteres, funciones completamente multiplicativas.

\begin{theorem}
Sea $G$ un grupo abeliano finito. Entonces

\begin{itemize}
    \item [(i)] Dado un carácter $\chi$ de $G$

$$
\sum_{g \in G} \chi(g)= \begin{cases}|G|, & \text { si } \chi=\chi_0 \\ 0, & \text { e.o.c }.\end{cases}
$$

\item[(ii)] Dado $g \in G$

$$
\sum_{\chi \in \widehat{G}} \chi(g)= \begin{cases}|G|, & \text { si } g=1 \\ 0, & \text { e.o.c }.\end{cases}
$$
\end{itemize}

\end{theorem}

\begin{proof}

Si $\chi=\chi_0$ entonces $\chi(g)=1$ para todo $g \in G$ por tanto la suma (i) es igual a $|G|$. Dado $\chi \neq \chi_0$, existe $h \in G$ tal que $\chi(h) \neq 1$, luego

$$
\chi(h) \sum_{g \in G} \chi(g)=\sum_{g \in G} \chi(h g)=\sum_{g \in G} \chi(g),
$$


ya que la  suma  se hace  sobre todos los elementos del grupo, $hg$ solo me permuta los elementos de la suma, así

$$(\chi(h)-1) \sum_{g\in G}\chi(g)=0,$$ 

y como $\chi(h) \neq 1$, obtenemos lo requerido. De manera  análoga obtenemos (ii), si $g=1$, entonces $\chi(g)=1$ para todo $\chi \in \widehat{G}$ y la suma (ii) es igual a $|\widehat{G}|=|G|$ por  el teorema \ref{dualidad}. Suponga que $g \neq 1$, entonces existe $\psi \in \widehat{G}$ tal que $\psi(g) \neq 1$. Nuevamente

$$
\psi(g) \sum_{\chi \in \widehat{G}} \chi(g)=\sum_{\chi \in \widehat{G}}(\psi \chi)(g)=\sum_{\chi \in \widehat{G}} \chi(g).
$$


De donde se sigue el resultado.\\
\end{proof}

\begin{corollary}[Ortogonalidad]

  Sea $G$ un grupo abeliano finito. Entonces

\begin{itemize}
    \item[(i)] Si $\chi$ y $\psi$ son caracteres de $G$

$$
\sum_{g \in G} \psi(g) \overline{\chi}(g)= \begin{cases}|G|, & \text { si } \psi=\chi ; \\ 0, & \text { e.o.c. }\end{cases}
$$

\item[(ii)] Si $g$ y $h$ son elementos de $G$

$$
\sum_{\chi \in \widehat{G}} \chi(g) \overline{\chi}(h)= \begin{cases}|G|, & \text { si } g=h \\ 0, & \text { e.o.c. }\end{cases}
$$
\end{itemize}

\end{corollary}

\begin{note}
Un lector familiarizado con algunos conceptos del análisis de Fourier podría notar que la suma presentada en (i) es muy similar al producto interno en $\ell^2(\Z)$, esto no es para nada coincidencia, en esencia son el mismo producto interno.\\

Sea $G$ un grupo abeliano finito se puede verificar que el espacio de funciones $ f:G \to \mathbb{C}$, es un espacio vectorial de dimensión finita con el producto interno

$$
\langle g, h \rangle = \sum_{g \in G} f(g) \overline{h(g)}.
$$

Además podemos ver a  $\widehat{G}$ como un subconjunto del espacio, ahora sí procedamos con la prueba.\\
\end{note}

\begin{proof}
La suma en (i) se puede escribir como $\displaystyle\sum_{g \in G}\left(\psi \chi^{-1}\right)(g)$ y

$$
\psi \chi^{-1}=\chi_0 \quad \text { si y solo si } \quad \psi=\chi.
$$


La suma en (ii) es igual a $\displaystyle\sum_{x \in \widehat{G}} \chi\left(g h^{-1}\right)$, luego el resultado es una consecuencia inmediata del teorema anterior.\\
\end{proof}

Procedamos formalmente suponiendo que tenemos 

$$
f(g)=\frac{1}{|G|}\sum_{\psi \in \widehat{G}}\widehat{f}(\psi) \psi(g),
$$

por la ortogonalidad

\begin{align*}
    \langle f, \chi\rangle&=\frac{1}{|G|}\sum_{g\in G}\sum_{\psi \in \widehat{G}}\widehat{f}(\psi) \psi(g)\overline{\chi}(g)\\
    &=\frac{1}{|G|}\sum_{\psi \in \widehat{G}}\widehat{f}(\psi)\sum_{g\in G}\psi(g)\overline{\chi}(g)\\
    &=\widehat{f}(\chi)
.\end{align*}

Esto motiva la siguiente definición de transformada de Fourier.

\begin{definition}[Transformada de Fourier
]
Sea $f: G \rightarrow \mathbb{C}$, definimos su transformada de Fourier como la función $\widehat{f}: \widehat{G} \rightarrow \mathbb{C}$ dada por

$$
\widehat{f}(\chi)=\sum_{g\in G} f(g) \overline{\chi}(g).
$$

\end{definition}
\begin{theorem}[Representación de Fourier
]
Dada $f: G \rightarrow \mathbb{C}$, tenemos la representación en ``serie'' de Fourier

$$
f(g)=\frac{1}{|G|}\sum_{\chi \in \widehat{G}}\widehat{f}(\chi) \chi(g).
$$

\end{theorem}

\begin{proof}
Dada $f:G\to \C$, por la ortogonalidad

\begin{align*}
    \frac{1}{|G|}\sum_{\chi \in \widehat{G}}\widehat{f}(\chi) \chi(g)&=\frac{1}{|G|}\sum_{\chi \in \widehat{G}}\sum_{h\in G} f(h) \overline{\chi}(h) \chi(g)\\
    &=\frac{1}{|G|}\sum_{h\in G} f(h) \sum_{\chi \in \widehat{G}}\overline{\chi}(h) \chi(g)\\
    &=f(g)
.\end{align*}

\end{proof}



\subsection{Caracteres de Dirichlet}

En esta sección nos restringimos a estudiar los caracteres del grupo $(\Z/q\Z)^{\times}$, sabemos que este es un grupo multiplicativo formado por las unidades del grupo aditivo $(\Z/q\Z)$, las clases de residuos módulo $q$, conocemos un par de cosas de este grupo, por ejemplo, su orden está determinado por el número de elementos de $(\Z/q\Z)$ que son primos relativos con $q$, esto es $\varphi(q)$.\\

Un carácter de Dirichlet es un carácter del grupo $(\Z/q\Z)^{\times}$, sin embargo es útil presentarlo como una función aritmética que extiende un carácter de $(\Z/q\Z)^{\times}$ a todos los naturales.

\begin{definition}[Carácter de Dirichlet]
Un carácter de Dirichlet es un carácter del grupo $(\mathbb{Z} / m \mathbb{Z})^{\times}$. Más generalmente, un carácter de Dirichlet módulo $m$ es una función aritmética que extiende un carácter de $(\mathbb{Z} / m \mathbb{Z})^{\times}$mediante

$$
f(n)= \begin{cases}\chi(a) & \text { si } n \equiv a (\bmod{m}), 1 \leq a \leq m, \text{ y }(a, m)=1, \\ 0 & \text { si }(n, m)>1.\end{cases}
$$

\end{definition}

Por ejemplo el carácter trivial se extiende de la  siguiente manera:

$$\chi_0(n)= \begin{cases}1, & \text { si }(n, m)=1 \\ 0, & \text { si }(n, m)>1\end{cases}.$$

En algunos casos simplificaremos la notación como antes, solo escribiremos si $(n,m)=1$ y no la condición $1\leq a\leq m$ y $n\equiv a (\bmod{m})$ ya que estas condiciones se pueden asumir que están presentes dado que estamos trabajando sobre clases de residuos, de todas manera es importante aclarar que solo es una herramienta de notación.

\begin{theorem}
Sea $\chi$ un carácter de Dirichlet módulo $q$, entonces
$\chi(m n)=\chi(m) \chi(n)$ y $\chi(n+q)=\chi(n)$ para todo $m, n \in \mathbb{N}$.
\end{theorem}

Esto es que en efecto cuando se extiende el carácter sobre todos los naturales, obtenemos una función periódica de periodo $q$ y  además esta sigue siendo completamente multiplicativa.\\

\begin{proof}
Sea $\chi$ un carácter de Dirichlet módulo $q$ y sean $m, n \in \mathbb{N}$. Sea $f$ el  carácter de $(\mathbb{Z} / q \mathbb{Z})^{\times}$ que induce $\chi$, esto es, para todo $n \in \mathbb{N}$,

$$
\chi(n)= \begin{cases}f(n), & \text { si }(n, q)=1 \\ 0, & \text { si }(n,q)>1\end{cases}
$$


si $(m, q)>1$ o $(n, q)>1$, entonces $(m n, q)>1$ y por tanto $\chi(m n)=0=$ $\chi(m) \chi(n)$. Si $(m, q)=(n, q)=1$, entonces $(m n, q)=1$ y

$$
\chi(m) \chi(n)=f(m) f(n)=f(mn)=\chi(m n) .
$$


Esto prueba que $\chi$ es completamente multiplicativa, tenemos que $(n+q, q)=(n, q)$ se sigue que si $(n+q, q)=1$, entonces $(n, q)=1$ y por tanto $\chi(n+q)=f(n+q)=$ $f(n)=\chi(n)$. Análogamente, si $(n+q, q)>1$, entonces $(n, q)>1$ y $\chi(n+q)=$ $0=\chi(n)$, entonces $\chi$ es periódica de periodo $q$.

\end{proof}

 Dada $f(n)$ la función característica de la progresión aritmética, es decir

$$
f(n)=\left\{\begin{array}{lll}
1, & n \equiv a & (\bmod{m} ) \\
0, & n \not\equiv a & (\bmod{m}),
\end{array}\right.
$$

entonces

$$
\hat{f}(\chi)=\sum_{g \in(\mathbb{Z} / m \mathbb{Z})^{\times}} f(g) \overline{\chi}(g)=\overline{\chi}(a)=\chi\left(a^{-1}\right),
$$

 ya que el término $f(g)$ es $0$ excepto cuando $g=a$ donde $f(g)$ vale 1, por lo tanto tenemos la representación de Fourier

$$
\boxed{f(n)=\frac{1}{\varphi(m)} \sum_\chi \chi\left(a^{-1}\right) \chi(n).
}
$$

Con este resultado hemos resuelto nuestro primer inconveniente para construir una prueba de teorema de Dirichlet. Observemos lo siguiente:

\begin{align*}
    \sum_{n=1}^{\infty}\frac{f(n)}{n^s}&=\frac{1}{\phi(m)}\sum_{n=1}^{\infty} \left(\frac{\displaystyle\sum_{\chi}\chi(a^{-1})\chi(n)}{n^s}\right)\\
    &=\frac{1}{\phi(m)}\sum_{\chi}\chi(a^{-1})\sum_{n=1}^{\infty}\frac{\chi(n)}{n^s}
,\end{align*}

en donde la serie de Dirichlet

$$\sum_{n=1}^{\infty} \frac{\chi(n)}{n^s}=L(s,\chi)$$

si resulta tener producto de Euler por lo que obtendremos la prueba de estudiar la expresión

\begin{equation}
\frac{1}{\phi(m)}\sum_{\chi}\chi(a^{-1})\log(L(1,\chi)).
\end{equation}

En la siguiente sección veremos que (2.2) es divergente y que en efecto es igual a la suma sobre los primos congruentes a $a (\bmod{m})$.

\section{La L-serie asociada a un carácter de Dirichlet}

En esta sección estudiaremos el comportamiento de las sumas parciales de $\chi(n)$, en particular nuestro objetivo es completar los ingredientes para la prueba del teorema de Dirichlet, por lo que en particular estudiaremos la no nulidad de 

$$\sum_{n=1}^{\infty} \frac{\chi(n)}{n}.$$

La razón de estudiar estas propiedades se esclarecerá con la prueba del teorema de Dirichlet, allí entenderemos la importancia de cada una de estas piezas que hemos ido construyendo.

\begin{definition}
Sea $\chi$ un carácter de Dirichlet módulo $q$, la $L$-serie de Dirichlet asociada a $\chi$ (también llamada $L$-función) es la serie (la función) 

$$
L(s, \chi)=\sum_{n=1}^{\infty} \frac{\chi(n)}{n^s} .
$$

\end{definition}

\begin{theorem}

$L(\chi,s)$ converge absolutamente si $\Re(s)>1$ y en este semiplano tenemos el producto de Euler

$$\sum_{n=1}^{\infty} \frac{\chi(n)}{n^s}=\prod_p \frac{1}{1-\chi(p) p^{-s}}.$$

\end{theorem}

\begin{proof}
Tenemos que $|\chi(n)|=1$, luego por la convergencia absoluta de $\zeta(s)$ en el semiplano $\Re(s)>1$ obtenemos la convergencia deseada, además como $\chi$ es completamente multiplicativa el producto de Euler se sigue del corolario 1.35.
\end{proof}

El teorema anterior nos da la siguiente expresión:

$$\begin{aligned}
L\left(s, \chi_0\right) & =\prod_p\left(1-\frac{\chi_0(p)}{p^s}\right)^{-1} \\
& =\prod_{p \nmid m}\left(1-\frac{1}{p^s}\right)^{-1},
\end{aligned}$$

para obtener una expresión precisa que podamos controlar sobre este producto una idea sería completar el producto sobre todos los primos y quitar los términos $p\mid m$ en otro producto, esto es:

$$
\begin{aligned}
L\left(s, \chi_0\right) & =\prod_p\left(1-\frac{\chi_0(p)}{p^s}\right)^{-1} \\
& =\prod_{p \nmid m}\left(1-\frac{1}{p^s}\right)^{-1} \\
& =\prod_{p \mid m}\left(1-\frac{1}{p^s}\right) \prod_p\left(1-\frac{1}{p^s}\right)^{-1} \\
& =\prod_{p \mid m}\left(1-\frac{1}{p^s}\right) \zeta(s) .
\end{aligned}
$$

La expresión $\displaystyle\prod_{p \mid m}\left(1-\frac{1}{p^s}\right)$ es continua cuando tomamos el límite $s\to 1$, a saber

$$\prod_{p \mid m}\left(1-\frac{1}{p}\right)=\frac{\varphi(m)}{m},$$

ya obtuvimos propiedades analíticas sobre la función $\zeta(s)$, sabemos que esta es analítica en el semiplano $\sigma>0$ excepto por un polo simple en $s=1$ con residuo 1 (teorema 1.40), esto nos da el siguiente corolario.

\begin{corollary}
$L\left(s, \chi_0\right)$ es una función analítica en $\Re(s)>0$, excepto por un polo simple en $s=1$ con residuo $\dfrac{\varphi(m)}{m}$.
\end{corollary}

Debemos obtener ahora propiedades analíticas sobre $L(s,\chi)$ con $\chi\neq \chi_0$, dado $\chi$ un carácter de Dirichlet módulo $q$, sea $A(x)=\displaystyle\sum_{n \leq x} \chi(n)$, como $\chi\neq \chi_0$, obtenemos que $\displaystyle\sum_{n=1}^q \chi(n)=0$, en general, $\displaystyle\sum_{n=1}^{k q} \chi(n)=0$ para todo $k \in \mathbb{N}$. Entonces

$$
|A(x)| \leq \sum_{n=1}^q|\chi(n)|=\sum_{\substack{n=1 \\(n, q)=1}}^q 1=\varphi(q)=O(1),
$$

y aplicando sumación parcial se sigue que

$$\begin{aligned}
L(s, \chi) & =\sum_{n=1}^{\infty} \frac{\chi(n)}{n^s}=\int_{1^{-}}^{\infty} \frac{1}{x^s} d A(x) \\
& =\left.\frac{A(x)}{x^s}\right|_{1^{-}} ^{\infty}+s \int_{1}^{\infty} \frac{A(x)}{x^s} \frac{d x}{x} \\
& =s \int_{1}^{\infty} \frac{A(x)}{x^{s+1}} d x ,
\end{aligned}$$



donde el término de borde se anula ya que $A(x)$ es $O(1)$ y en el límite inferior las sumas parciales son 0. Note que la expresión anterior es válida para $\Re(s)>0$ dado que $A(x)=O(1)$, además ya vimos que la expresión

$$\int_1^{\infty}\frac{1}{x^{s+1}}dx$$

es analítica en el semiplano $\sigma>0$ (Teorema 1.40), obtenemos el siguiente corolario
\begin{corollary}
Si $\chi \neq \chi_0, L(s, \chi)$ es una función analítica en $\Re(s)>0$. (no tiene polos)
\end{corollary}

\subsection{Prueba del teorema de Dirichlet}

\begin{theorem}[Dirichlet]
Sea $\chi \neq \chi_0$, entonces \[
\log L(1, \chi) \neq 0.
\]
\end{theorem}

Omitiremos la prueba de este teorema hasta completar la prueba del teorema de Dirichlet.


\begin{theorem}[Dirichlet 1837]
Sean $(a, m) = 1$, entonces existen infinitos primos en la progresión aritmética: $a, a + m, a + 2m, a + 3m, a + 4m, \dots$

\end{theorem}

\begin{proof}
Imitando las ideas de Euler tenemos para $\Re(s) > 1$
\[
\log L(s, \chi) = \sum_p \sum_{n=1}^{\infty} \frac{\chi(p)}{np^{ns}} = \sum_p \frac{\chi(p)}{p^s} + \sum_{n=2}^{\infty} \sum_p \frac{\chi(p)}{np^{ns}} = \sum_p \frac{\chi(p)}{p^s} + O(1).
\]

Ahora usando que
\[f(n)=
\frac{1}{\varphi(m)} \sum_{\chi (a^{-1})} \chi(a^{-1}) \chi(n) = 
\begin{cases}
1, & n \equiv a \pmod{m} \\
0, & n \not\equiv a \pmod{m}
\end{cases}
\]
tenemos lo siguiente:

\begin{align*}
    \frac{1}{\varphi(m)} \sum_{\chi} \chi(a^{-1}) \log L(s, \chi)&= \frac{1}{\varphi(m)} \sum_{\chi} \chi(a^{-1})\left(\sum_p \frac{\chi(p)}{p^s} + O(1)\right)\\
    &= \sum_p \frac{1}{p^s} \left( \frac{1}{\varphi(m)} \sum_{\chi} \chi(a^{-1}) \chi(p) \right) + O(1)\\
&= \sum_{\substack{p}} \frac{f(p)}{p^s} + O(1)\\
&=\sum_{\substack{p \equiv a(m)}} \frac{1}{p^s} + O(1)
.\end{align*}

Así para mostrar que la serie diverge, y por lo tanto que existen infinitos primos en la progresión aritmética, resta ver que
\[
\lim_{s \to 1^+} \frac{1}{\varphi(m)} \sum_{\chi (a^{-1})} \log L(s, \chi) = +\infty,
\]

ya sabemos que
\[
\lim_{s \to 1^+} \log L(s, \chi_0) = +\infty,
\]
por el teorema anterior tenemos que para $\chi \neq \chi_0$
\[
\lim_{s \to 1^+} \log L(s, \chi)
\]
es finito ya que $L(1,\chi)\neq 0$ y $L(s,\chi)$ no tiene polos en $\Re(s)>0$, entonces el logaritmo no explota, lo anterior nos da que 

\begin{align*}
     +\infty&=\lim_{s \to 1^{+}} \frac{1}{\varphi(m)} \sum_{\chi} \chi(a^{-1}) \log L(s, \chi)\\
     &=\sum_{\substack{p \equiv a(m)}} \frac{1}{p} + O(1)
.\end{align*}
\end{proof}

La idea que tuvo Dirichlet de usar caracteres para escribir la función característica como combinación lineal de funciones completamente multiplicativas quizás venga de que Dirichlet no estudió en Alemania, estudió en Francia bajo supervisión de varios matemáticos, entre ellos Fourier, por lo que venía con algunas ideas de representación de funciones, ideas que estaban surgiendo del estudio de la ecuación del calor, esta es la razón por la que se considera a este resultado como el nacimiento de la teoría analítica de números, conecta dos áreas de la matemática aparentemente alejadas.\\

La primera prueba que presentó Dirichlet estaba incompleta, solo abordaba el caso en el que $m$ era un número primo, para el caso general tuvo que asumir su fórmula del número de clases que demostró en un artículo en 1839-1840, varios años luego de su primera prueba, al final de su artículo menciona que la primera prueba que se le ocurrió del resultado vital que necesitaba (la no nulidad de $L(1,\chi)$, $\chi\neq\chi_0$) venía de  argumentos más indirectos y complicados, dice ``no creo que haya un indicio en algún lugar de su naturaleza''.\\

\begin{note}
La prueba original de \(L(1, \chi) \neq 0\) para un carácter real \(\chi\) viene de asociar este carácter al símbolo de Legendre y es una simple consecuencia de su fórmula del número de clases. Sea \(K = \mathbb{Q}(\sqrt{D})\) un extensión cuadrática con \(D \equiv 0, 1 \pmod{4}\), existe un carácter primitivo real asociado \(\chi\) tal que
\[
\zeta_K(s) = \sum_a (N\mathfrak{a})^{-s} = \zeta(s)L(s, \chi),
\]
y cada carácter real primitivo módulo \( q \neq 1 \) se obtiene de esta forma a partir de \( K = \mathbb{Q}(\sqrt{\chi(-1)q}) \). Dirichlet demostró que
\[
L(1, \chi) = 
\begin{cases}
\dfrac{2\pi h}{w \sqrt{q}} & \text{si } \chi(-1) = -1, \\[8pt]
\dfrac{2h \log \varepsilon}{\sqrt{q}} & \text{si } \chi(-1) = 1,
\end{cases}
\]
donde \(h\) es el número de clases de \(K\), \(w\) es el número de unidades en el anillo de enteros de \(K\) si \(\chi(-1) = -1\), y \(\varepsilon\) es la unidad fundamental de \(K\) si \(\chi(-1) = 1\). Por lo tanto $L(1, \chi) > 1/\sqrt{q}$.\quad \cite{iwaniec2021analytic}\\
\end{note}

\section{La no nulidad de \texorpdfstring{$L(1,\chi)$}{Lg}}

En esta sección dividiremos la prueba en dos partes, primero veremos la no nulidad de $L(1,\chi)$ para $\chi\neq\chi_0$ un carácter real, luego la no nulidad para un carácter a valor complejo, haremos esto al igual que Dirichlet puesto que es más sencillo trabajar los dos casos por separado, toda nuestra atención estará al inicio sobre el caso del carácter real dado que es el caso más complicado, sin embargo la prueba que presentaremos no es la de Dirichlet, es posible obtener este resultado con el método de Dirichlet de la hipérbola que estudiamos en el capítulo 1. Dado un carácter real $\chi\neq\chi_0$ módulo $n$ considere la función

$$A(n)=\sum_{j\mid n}\chi(j),$$

la  pregunta natural es ¿cuál es la serie de Dirichlet asociada a $A(n)$?, note que $A(n)=(\chi*1)(n)$, entonces

$$\sum_{n=1}^{\infty} \frac{A(n)}{n^s}=\sum_{n=1}^{\infty} \frac{(\chi*1)(n)}{n^s}=\sum_{n=1}^{\infty} \frac{\chi(n)}{n^s}\sum_{n=1}^{\infty} \frac{1}{n^s}=L(s,\chi)\zeta(s),$$

es posible ver que esta expresión es justamente $\zeta_K(s)$ donde $\zeta_K(s)$ es la función zeta de Dedekind en la extensión cuadrática $K=\Q(\sqrt{D})$ con $D$ el entero positivo mayor que 1 más pequeño que divide a $n$.\\

En 1895 Mertens presentó una prueba de este teorema haciendo estimaciones sobre la función 

$$B(x)=\sum_{n\leq x}\frac{A(n)}{\sqrt{n}}=\sum_{n\leq x}\sum_{d\mid n}\frac{\chi(d)}{\sqrt{n} },$$

demostró que $B(x)=2 \sqrt{x}L(1,\chi)+O(1)$ y que $B(x)$ diverge cuando tomamos $x\to\infty$, por lo tanto $L(1,\chi)\neq 0$, para ver esto utiliza el método de Dirichlet de la hipérbola dado $B(x)$ son las sumas parciales de una convolución. La prueba que presentaremos es la de Landau salvo por unas pequeñas modificaciones, las ventaja es que esta demostración es más sencilla y autocontenida que la de Dirichlet, no requerimos herramientas de la teoría algebraica de números, aunque como vimos la idea subyacente proviene de allí.

\begin{theorem}
Sea $\chi=\chi_0$ un carácter módulo $k$ y sea $f$ una función no negativa con derivada continua tal que $f^{\prime}(x)<0$ para $x\geq x_0$. Entonces si $y\geq x\geq x_0$, tenemos que

$$\sum_{y<n\leq x}\chi(n)f(n)=O(f(x)),$$

si adicionalmente $f(x)\to 0$ cuando $x\to \infty$ entonces la serie

$$\sum_{n=1}^{\infty} \chi(n)f(n)$$

converge y tenemos que para $x\geq x_0$

$$\sum_{n\leq x}\chi(n)f(n)=\sum_{n=1}^{\infty} \chi(n)f(n)+O(f(x))$$
\end{theorem}

\begin{proof}
Tenemos que $A(x)=\displaystyle \sum_{n\leq x}\chi(n)=O(1)$, por lo tanto 

$$
\begin{aligned}
\sum_{x< n \leqslant y} f(n) f(n) & =\int_x^y f(t) d(A(t)) \\
& =f(y) A(y)-f(x) A(x)-\int_x^y A(t) f^{\prime}(t) d t \\
& \ll f(y)+f(x)+\int_x^y\left(-f^{\prime}(t)\right) d t \\
& \ll f(x),
\end{aligned}
$$

si adicionalmente $f(x) \rightarrow 0$ cuando $x \rightarrow \infty$, $\displaystyle\sum_{x \leqslant n \leqslant y} x(n) f(n) \ll f(x) \rightarrow 0$ cuando $x, y \rightarrow \infty$ esto es $\displaystyle\sum_{n=1}^{\infty} x(n) f(n)$ converge por el criterio de Cauchy. Finalmente note que 

$$
\begin{aligned}
\sum_{n=1}^{\infty} x(n) f(n) & =\sum_{n \leqslant x} x(n) f(n)+\lim _{y \rightarrow \infty} \sum_{x<n \leqslant y} x(n) f(n) \\
&=\sum_{n \leqslant x} x(n) f(n)+O(f(x))
\end{aligned}$$
\end{proof}

Aplicando este teorema para $f(x)=\dfrac{1}{x}$, $f(x)=\dfrac{\log x}{x}$ y $f(x)=\dfrac{1}{\sqrt{x} }$ obtenemos el siguiente corolario

\begin{corollary}
Si $\chi\neq\chi_0$ es un carácter módulo $k$ y si $x\geq 1$, tenemos que

$$\begin{aligned}
\sum_{n \leqslant x} \frac{\chi(n)}{n} & =L(1, \chi)+O\left(\frac{1}{x}\right) \\
\sum_{n \leqslant x} \frac{\chi(n) \log n}{n} & =\sum_{n=1}^{\infty} \frac{\chi(n) \log n}{n}+O\left(\frac{\log x}{x}\right) \\
\sum_{n \leqslant x} \frac{\chi(n)}{\sqrt{n}} & =L\left(\frac{1}{2}, \chi\right)+O\left(\frac{1}{\sqrt{x}}\right) .
\end{aligned}$$
\end{corollary}

\subsection{No nulidad para el carácter real \texorpdfstring{$\chi\neq\chi_0$}{Lg}}


\begin{theorem}
Sea $\chi$ un carácter real módulo $k$ y considere la función

$$A(n)=\sum_{d\mid n} \chi(n).$$

Entonces $A(n)\geq 0$ para todo $n$ y $A(n)\geq 1$ si $n$ es un cuadrado perfecto.
\end{theorem}

\begin{proof}
Para las potencias de primos tenemos que

$$A(p^a)=\sum_{t=0}^{a}\chi(p^t)=1+\sum_{t=1}^{a} \chi(p)^t,$$

como $\chi$ es a valor real, entonces los único valores de $\chi(p)$  son $0,1$ y $-1$, si $\chi(p)=0$, entonces $A(p^a)=1$, si $\chi(p)=1$ entonces $A(p^a)=a+1$, si $\chi(p)=-1$ entonces

$$A(p^a)=\begin{cases}
0 \quad\text{ si } a \text{ es impar},\\
1 \quad\text{ si } a \text{ es par}.
\end{cases}$$

En cualquier caso, $A(p^a)\geq 1$ si $a$ es par. Ahora, si $n=p_1^{a_1}\cdot\cdot\cdot p_r^{a_r}$ entonces $A(n)=A(p_1^{a_1})\ldots A(p_r^{a_r})$ ya que $A$ es multiplicativa, dado que es convolución de dos funciones multiplicativas. Cada factor $A(p_i^{a_i})\geq 0$ de esto se sigue que $A(n)\geq 0$, además si $n$ es un cuadrado perfecto, entonces cada exponente $a_i$ es par, esto es que cada factor $A(p_i^{a_i})\geq 1$ por lo cual $A(n)\geq 1$.
\end{proof}


\begin{lemma}[Sumas parciales de $\zeta(s)$]
Dado $x\geq 1$, $s>0$ y $s\neq 1$ tenemos que

$$
\sum_{n \leq x} \frac{1}{n^s}=\frac{x^{1-s}}{1-s}+\zeta(s)+O\left(x^{-s}\right)
$$
\end{lemma}

\begin{proof}
Note que 

\begin{align*}
    \sum_{n\leq x}\frac{1}{n^s}&=1+\int_1^x\frac{1}{t^s}d[t]\\
    &=1+\int_1^x\frac{1}{t^s}dt-\int_1^x\frac{1}{t^s}d\{t\}\\
    &=1+\frac{x^{1-s}}{1-s}-\frac{1}{1-s}-\int_1^x\frac{1}{t^s}d\{t\}\\
    &=1+\frac{x^{1-s}}{1-s}-\frac{1}{1-s}-\left(\frac{\{x\}}{x^s}+s\int_1^xt^{-s-1}\{t\}dt\right)  
.\end{align*}

Como $\{t\}=O(1)$, entonces 

\begin{align*}
    \sum_{n\leq x}\frac{1}{n^s}&=\frac{x^{1-s}}{1-s}+1+\frac{1}{s-1}-s\int_1^x\frac{\{t\}}{t^{s+1}}dt+O(x^{-s})\\
    &=\frac{x^{1-s}}{1-s}+\frac{s}{s-1}-s\int_1^x \frac{\{t\}}{t^{s+1}}dt+O(x^{-s})\\
    &=\frac{x^{1-s}}{1-s}+\zeta(s)+O(x^{-s})
,\end{align*}

por la representación integral de $\zeta(s)$, esta representación integral tiene sentido para $s>0$ con $s\neq 1$, lo que concluye el resultado.
\end{proof}


\begin{theorem}
Dado $\chi\neq\chi_0$ un carácter real módulo $k$, sea

$$A(n)=\sum_{d\mid n}\chi(d)\quad \text{y} \quad B(x)=\sum_{n\leq x}\frac{A(n)}{\sqrt{n}}.$$

Tenemos que

\begin{itemize}
\item[i)] $B(x)\to\infty$ cuando $x\to\infty$

\item[ii)] $B(x)=2 \sqrt{x} L(1.\chi)+O(1)$
\end{itemize}
\end{theorem}

\begin{proof}
Tenemos que

$$B(x)\geq \sum_{\substack{n\leq x\\
n=m^2}}\frac{1}{\sqrt{n}}=\sum_{n\leq x} \frac{1}{m} $$

y por lo tanto cuando $x\to \infty$ la suma diverge por la divergencia de la serie armónica, esto prueba $i)$. Note que cambiando el orden de sumación

\begin{align*}
    B(x) &= \sum_{n \leq x} \sum_{\substack{d \mid n}} \frac{\chi(d)}{\sqrt{n}} 
     = \sum_{d \leq x} \sum_{\substack{n \leq x \\ d \mid n}} \frac{\chi(d)}{\sqrt{n}}\\
     &= \sum_{d \leq x} \chi(d) \sum_{\substack{n \leq x \\ d \mid n}} \frac{1}{\sqrt{n}}\\
     &= \sum_{d \leq x} \chi(d) \sum_{k \leq \frac{x}{d}} \frac{1}{\sqrt{kd}}\\
     &=\sum_{dk \leq x} \frac{\chi(d)}{\sqrt{kd}}
.\end{align*}

Dado que la última expresión para $B(x)$ son sumas parciales de una convolución, podemos aplicar el método de Dirichlet de hipérbola como sigue

$$\begin{aligned}
B(x)& =\sum_{d \leqslant \sqrt{x}} \frac{x(d)}{\sqrt{d}} \sum_{k \leq\frac{x}{d}} \frac{1}{\sqrt{k}}+\sum_{k \leqslant \sqrt{x}} \frac{1}{\sqrt{k}} \sum_{d\leq x / k} \frac{x(d)}{\sqrt{d}}-\sum_{k \leqslant \sqrt{x}} \frac{1}{\sqrt{k}} \sum_{d \leqslant \sqrt{k}} \frac{x(d)}{\sqrt{d}} \\
& =S_1+S_2-S_3,
\end{aligned}$$


Vamos a trabajar por separado las expresiones $S_1, S_2$ y $S_3$, note que podemos estimar $\displaystyle \sum_{k\leq \frac{x}{d}} \frac{1}{\sqrt{k}}$ aplicando  el teorema anterior y tomaremos $a=\zeta(s)$ dado que para este valor de $s$ la suma converge, así obtenemos

\begin{align*}
S_1 &= \sum_{d \leq \sqrt{x}} \frac{\chi(d)}{d} 
\left( 2 \left( \frac{x}{d} \right)^{1/2} +a + O\left( \left( \frac{d}{x} \right)^{1/2} \right) \right) \\
&= 2\sqrt{x} \sum_{d \leq \sqrt{x}} \frac{\chi(d)}{d} 
+ a \sum_{d \leq \sqrt{x}} \frac{\chi(d)}{\sqrt{d}} 
+ O\left( \frac{1}{\sqrt{x}} \sum_{d \leq \sqrt{x}} |\chi(d)| \right) \\
&= 2\sqrt{x} \left( L(1, \chi) + O(x^{1/2}) \right) 
+ a L\left(\frac{1}{2}, \chi\right) + O\left(x^{-1/4}\right) + O(1) \\
&= 2\sqrt{x} L(1, \chi) + a L\left(\frac{1}{2}, \chi\right) + O(1) \\
&= 2\sqrt{x} L(1, \chi) + O(1).
\end{align*}

Análogamente estimamos la suma $\displaystyle \sum_{k\leq \sqrt{x}} \frac{1}{\sqrt{k} }$

$$
\begin{aligned}
S_2 & =\sum_{k \leq \sqrt{x}} \frac{1}{\sqrt{k}}\left(L(1 / 2, \chi)+O\left(\frac{\sqrt{k} }{\sqrt{x}}\right)\right) \\
& =L(1 / 2, \chi)\left(2 x^{1 / 4}+O(1)\right)+O\left(\sum_{k \leq \sqrt{x}} \frac{1}{\sqrt{x}}\right) \\
& =L(1 / 2, \chi) 2 x^{1 / 4}+O(1)
\end{aligned}
$$

y


\begin{align*}
    S_3 & =\left(L(1 / 2, \chi)+O\left(\frac{1}{x^{1 / 4}}\right)\right)\left(2 x^{1 / 4}+O(1)\right) \\
& =L(1 / 2, \chi) 2 x^{1 / 4}+O(1),
\end{align*}

por lo tanto $B(x)=S_1+S_2-S_3=2 \sqrt{x}L(1,\chi)+O(1)$ ya que los términos $L(\frac{1}{2},\chi)2x^{\frac{1}{4}}$ se cancelan.

\end{proof}

Como $B(x)$ diverge cuando $x\to\infty$ entonces $L(1,\chi)\neq0$

\subsection{No nulidad del carácter complejo \texorpdfstring{$\chi\neq\chi_0$}{Lg}}

Considere la función

$$P(s)=\prod_{\chi}L(s,\chi),$$

donde $\chi$ es un carácter de Dirichlet módulo $q$, note que 

$$P(s)=L(\chi_0,s)\prod_{\chi\neq \chi_0}L(s,\chi),$$

sabemos que este último producto está acotado en $\Re(s)>0$ por el corolario 2.13 y que la función $L(\chi, s)$ tiene un polo simple en $s=1$ dado por $\zeta(s)$, el teorema 1.40 nos dice más aún, nos da que cuando $s\to 1^{+}$, $L(s,\chi_0)=O\left(\dfrac{1}{s-1}\right)$, Supongamos que $L(1,\chi_1)=0$ para algún carácter $x_1\neq\chi_0$, como $\chi_1$ es un carácter complejo, entonces $\chi_1\neq \overline{\chi_1}$, además

$$L(s,\overline{\chi_1})=\sum_{n=1}^{\infty} \frac{\overline{\chi_1}(n)}{n^s}=\overline{\sum_{n=1}^{\infty} \frac{\chi_1(n)}{n^s}}=\overline{L(s,\chi_1)}$$

de esto se sigue que si $L(1,\chi_1)=0$, entonces $L(1,\overline{\chi_1})=0$, estas consideraciones nos permiten presentar una prueba sencilla de la no nulidad de $L(1,\chi)$, note que fue necesario separar el caso del carácter real dado que si $\chi_1$ es un carácter real entonces $\chi_1=\overline{\chi_1}$ y esto daña el argumento.

\begin{lemma}
Sea $P(s)=\displaystyle\prod_{\chi}L(s,\chi)$, si $s>1$ entonces $P(s)\geq 1$.
\end{lemma}

\begin{proof}
Tenemos que

$$\begin{aligned}
\log (P(s)) & =\sum_{\chi} \log L(s, \chi)=\sum_{\chi \bmod q} \sum_p \log \left(1-\frac{\chi(p)}{p^s}\right)^{-1} \\
& =\sum_{\chi} \sum_p \sum_{m=1}^{\infty} \frac{\chi(p)^m}{mp^{m s}} \\
& =\sum_p \sum_{m=1}^{\infty} \frac{1}{mp^{ms}} \sum_{\chi} \chi(p)^m .
\end{aligned},$$

por la ortogonalidad de los caracteres tenemos que

$$\sum_{\chi} \chi(p)^m=\sum_{\chi}\overline{\chi}(1) \chi\left(p^m\right)= \begin{cases}\phi(q) & \text { si } p^m \equiv 1 (\bmod{q}), \\ 0 & \text { e.o.c, }\end{cases}$$


esto nos da que $\displaystyle\sum_p \sum_{m=1}^{\infty} \frac{1}{mp^{ms}} \sum_{\chi} \chi(p)^m$ es un suma de términos no negativos, de esto se sigue que

$$P(s)=exp\left(\sum_p \sum_{m=1}^{\infty} \frac{1}{mp^{ms}} \sum_{\chi} \chi(p)^m\right)\geq 1$$
\end{proof}

\begin{theorem}
Sea $\chi\neq \chi_0$ un carácter complejo módulo $q$, entonces $L(1,\chi)\neq 0$.
\end{theorem}

\begin{proof}
Supongamos que existe $\chi_1\neq \chi_0$ tal que $L(1,\chi_1)=0$, entonces $L(1,\overline{\chi_1})=0$, por lo tanto cuando $s\to 1^{+}$ tenemos que $L(s,\chi_1)=O(s-1)=L(s,\overline{\chi_1})$, en efecto

$$P(s)=L(s,\chi_0)L(s,\chi_1)L(s,\overline{\chi_1})Q(s),$$

con $$Q(s)=\prod_{\chi\neq \chi_1,\overline{\chi_1},\chi_0}L(s,\chi)=O(1), \quad \Re(s)>0.$$


Al tomar el $\displaystyle \lim_{s \to 1^{+}} P(s)$ basta controlar los otros 3 factores, pero ya vimos que $L(s,\chi_1)=O(s-1)=L(s,\overline{\chi_1})$, por lo tanto $P(s)=O(s-1)$ cuando $s\to 1^{+}$, esto contradice el lema 2.21.\\
\end{proof}

Sea $s>0$, en esencia lo que ocurre en la prueba anterior es que los ceros de $L(s,\chi)$, con $\chi\neq \chi_0$ vienen en parejas por lo que si $L(s,\chi)=0$, entonces $L(s,\overline{\chi})=0$, esto es conveniente para este caso, un cero nos anula el efecto del polo dado por el carácter trivial y el otro nos da la contradicción.