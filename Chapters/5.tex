%!TEX root = ../main.tex

\thispagestyle{empty}
\vspace{-0.5cm}

\cleanchapterquote{Los encantos de esta ciencia sublime, las matemáticas, solo se le revelan a aquellos que tienen el valor de profundizar en ella.}{Carl Friedrich Gauss}{}

El teorema de Dirichlet nos dice que hay infinitos primos de la forma $a,a+n , a+2n,\ldots$ siempre que $(a,n)=1$, este es un resultado importante en teoría de números sin embargo podemos decir mucho más sobre primos en progresiones aritmética. Hemos probado el teorema de los números primos en el capítulo anterior y una pregunta natural es cómo se extiende este a progresiones aritmética, por ejemplo, si estudiamos los primos de la forma $4k+1$ y $4k+3$, ¿hay la misma cantidad de primos de la forma $4k+1$ que $4k+3$?, ¿podemos esperar una distribución uniforme de estos?, la respuesta nos la dará el TNP, este nos dice que

$$\pi(1,4,x)\thicksim \frac{x}{\varphi(4)\log x}=\frac{x}{2\log x},\quad \pi(3,4,x)\thicksim \frac{x}{2\log x}.$$

Más precisamente $(\Z/4\Z)^\times=\{\overline{1},\overline{3}\}$, tenemos dos clases generadoras de primos y en cada una tenemos la mitad, este resultado no es un hecho trivial, por mucho tiempo los matemáticos pensaron que este no era el caso, la evidencia heurística apuntaba a que habían más primos de la forma $4k+3$ hasta que Littlewood demostró que $\pi(3,4,x)-\pi(1,4,x)$ tiene infinitos cambios de signo, estos temas se abordan de manera más detallada en \cite{granville2006prime}.
\section{Teorema de los números primos en progresiones aritmética}

Como antes, la idea es aplicar el teorema de Wiener-Ikehara para probar el TNP, por lo que presentamos antes este teorema sobre no nulidad de $L-$funciones, análogo al que tenemos para $\zeta(s)$

\begin{theorem}
    Sea $\chi$ un carácter de Dirichlet, entonces

    $$L(\chi,s)\neq 0, \text{ si }\Re(s)>1$$
\end{theorem}

\begin{proof}
    Como $\Re(s)>1$, entonces

    $$L(\chi,s)=\prod_p\frac{1}{1-\chi(p)p^{-s}},$$

    de manera análoga al teorema 1.38, el término $\dfrac{1}{1-\chi(p)p^{-s}}=\dfrac{p^s}{p^s-\chi(p)}\neq 0$ para todo $p$, esto nos  da que el producto no tiene factores nulos y por tanto no converge a 0.
\end{proof}

\begin{theorem}[Teorema de los números primos en progresiones aritmética]
Dados $a$ y $m$ primos relativos.
    $$\pi(a,m,x)\thicksim \frac{x}{\varphi(m)\log x}$$    
\end{theorem}

Observemos lo siguiente, si $h$ es una función aritmética completamente multiplicativa

\begin{align*}
    ((f*g)h)(n)&=\sum_{j\mid n}f(j)g\left(\frac{n}{j}\right)h(n)=\sum_{j\mid n}f(j)g\left(\frac{n}{j}\right)h\left(j\right)h\left(\frac{n}{j}\right)\\
    &=\sum_{j\mid n}f(j)h\left(j\right)g\left(\frac{n}{j}\right)h\left(\frac{n}{j}\right)\\
    &=\sum_{j\mid n}fh(j)gh\left(\frac{n}{j}\right)\\
    &=(fh*gh)(n)
.\end{align*}

Podemos aplicar esto para obtener una fórmula para la derivada logarítmica de la función $L(\chi,s)$

\begin{align*}
    -L^{\prime}(\chi,s)&=\sum_{n=1}^{\infty} \frac{\chi(n)\log(n)}{n^s}=\sum_{n=1}^{\infty} \frac{\chi(n)(\Lambda*1)(n)}{n^s}=\sum_{n=1}^{\infty} \frac{(\chi\Lambda*1\chi)(n) }{n^s}\\
    &=\sum_{n=1}^{\infty} \frac{\chi(n)\Lambda(n)}{n^s}\sum_{n=1}^{\infty} \frac{\chi(n)}{n^s},
\end{align*}

esto es

$$\boxed{-\frac{L^{\prime}}{L}(\chi,s)=\sum_{n=1}^{\infty} \frac{\chi(n)\Lambda(n)}{n^s}}.$$

Obtendremos el TNP en progresiones aritmética de aplicar el teorema de Wiener Ikehara a la función

$$F(s)=\sum_{n\equiv a \bmod{m}}\frac{\Lambda(n)}{n^s},$$

pero para esto debemos primero ver qué forma tiene esta función, sea $f(n)$ la función característica de la progresión, usando su representación de Fourier obtenemos que

\begin{align*}
    F(s)&=\sum_{n=1}^{\infty}\frac{\Lambda(n)f(n)}{n^s}=\sum_{n=1}^{\infty} \frac{\Lambda(n)}{n^s}\left(\frac{1}{\varphi(m)}\sum_\chi \chi(a^{-1})\chi(n)\right)\\
    &=\frac{1}{\varphi(m)}\sum_\chi\chi(a^{-1})\left(\sum_{n=1}^{\infty} \frac{\chi(n)\Lambda(n)}{n^s}\right)\\
    &=\frac{1}{\varphi(m)}\sum_\chi\chi(a^{-1})\left(-\frac{L^{\prime}}{L}(\chi,s)\right)
.\end{align*}

Sabemos que todas estas $L-$funciones presentes en la suma son analíticas en $\Re(s)\geq 1$ excepto $L(\chi_0,s)$ que tiene un polo simple en $s=1$ dado por el polo simple que tiene $\zeta(s)$, aplicando el principio del argumento y suponiendo que ninguna de estas $L$ funciones se anula en la recta vertical $\Re(s)=1$ sabremos que en la suma anterior el único carácter que contribuye un polo es el trivial, este polo tendrá residuo $1$, luego la función $F(s)$ cumple todas las condiciones del teorema de Wiener Ikehara y además tiene residuo $\dfrac{1}{\varphi(m)}$, esto es

$$\psi(a,m,x)=\sum_{\substack{n\leq x\\n\equiv a\bmod{m}}}\Lambda(n)\thicksim \frac{x}{\varphi(m)}.$$

Repasando lo anterior, las derivadas logarítmicas son todas analíticas en $\Re(s)\geq 1$ por el principio del argumento, solo una tiene un polo en $s=1$, la del carácter trivial y esta nos da el residuo que esperamos. El teorema de los números primos es una consecuencia inmediata de la siguiente afirmación
\begin{theorem}
    Sea $\chi$ un carácter de Dirichlet, $L(\chi,1+it)\neq 0$  para todo $t$
\end{theorem}

La prueba de este teorema no es sencilla, recordemos que para $\zeta(s)$ no lo fue, requería de ciertas estimaciones que no son evidentes, en general estudiar las regiones de no nulidad de $L-$funciones es un problema profundo en teoría analítica de números que además permanece abierto. La prueba que veremos aquí es la propuesta en \cite{murty2007problems}.

\subsection{La no nulidad de \texorpdfstring{$L(1+it,\chi)$ }{Lg}}

Como en el capítulo anterior, probar esta no nulidad requiere de identidades trigonométricas, ya que al tomarle parte real a una serie de Dirichlet nos aparecen cosenos de manera natural, sin embargo, cuando vimos que $L(1,\chi)\neq 0$ para $\chi\neq\chi_0$ consideramos un producto de $L-$funciones y realizamos un argumento de conteo de ceros y polos. En este caso también vamos a considerar el producto de todas las $L$-funciones, veremos que

$$f(s)=\prod_\chi L(s,\chi)$$ 

no se anula en la recta vertical $\Re(s)=1$ y por lo tanto ninguna $L-$función, para esto aplicaremos un argumento que proviene de Landau y que es modificado en \cite{murty2007problems}, el estudio de las series de Dirichlet con coeficientes no negativos, por lo que estudiaremos la expresión $\log(f(s))$.\\

 Finalmente, como estamos trabajando con $\log(f(s))$, el logaritmo transforma este producto de $L-$funciones en una suma y vamos a necesitar identidades sobre suma de cosenos, particularmente la identidad de Lagrange para el núcleo de Dirichlet.

 \begin{theorem}
Para $0 < \theta < 2\pi$.
     \[
    \frac{1}{2} + \cos \theta + \cos(2\theta) + \dots + \cos(n\theta) = \frac{\sin\left((n + \frac{1}{2})\theta\right)}{2 \sin\left(\frac{\theta}{2}\right)},
    \]
 \end{theorem}

 \begin{proof}
     Note que si tomamos $|z|=1$ con $z\neq \pm 1$, entonces $z=exp(i\theta)$, de esto se sigue que

    $$\Re\left(\sum_{k=0}^{n} z^k\right)=\sum_{k=0}^{n} \Re(z^k)=\sum_{k=0}^{n}cos(k\theta).$$

    En efecto
    \begin{align*}
      \Re\left(\sum_{k=0}^{n} z^k\right)&=\Re\left(\frac{z^{n+1}-1}{z-1}\right)\\
      &=\Re\left(\frac{exp\left(i(n+1)\theta)\right) -1}{exp(i\theta)-1}\right)\\
      &=\Re\left(\frac{exp(i(n+1)\theta/2)}{exp(i\theta/2)}\frac{exp(-i(n+1)\theta/2)-exp(i(n+1)\theta/2)}{exp(-i\theta/2)-exp(i\theta/2)}\right)
    .\end{align*}

   Note que el último término de la derecha  de puede escribir como

   $$\Re\left(exp(in\theta/2)\frac{\sin\left((n+1)\theta/2\right)}{\sin(\theta/2)}\right)=\cos(n\theta/2)\frac{\sin((n+1)\theta/2)}{\sin(\theta/2)},$$

   además
   
   $$
\begin{aligned}
\cos \left(\frac{n \theta}{2}\right) \sin \left(\frac{(n+1) \theta}{2}\right)&=\frac{1}{2}\left(\sin \left(\frac{\theta n}{2}+\frac{\theta(n+1)}{2}\right)+\sin \left(\frac{\theta(n+1)}{2}-\frac{n \theta}{2}\right)\right. \\
& =\frac{1}{2}\left(\sin \left(\frac{\theta(2 n+1)}{2}\right)+\sin \left(\frac{\theta}{2}\right)\right) \\
& =\frac{1}{2} \sin \left(\theta\left(n+\frac{1}{2}\right)\right)+\frac{1}{2} \sin \left(\frac{\theta}{2}\right),
\end{aligned}
$$
dividiendo entre $\sin\left(\displaystyle\frac{\theta}{2}\right)$ obtenemos que

\[
    1 + \cos \theta + \cos(2\theta) + \dots + \cos(n\theta) = \frac{1}{2} + \frac{\sin\left((n + \frac{1}{2})\theta\right)}{2 \sin\left(\frac{\theta}{2}\right)},
    \]

de lo que se sigue el resultado.
 \end{proof}

 \begin{lemma}
$$1+\frac{\sin 3 \theta}{\sin \theta}+\frac{\sin 5 \theta}{\sin \theta}+\cdots+\frac{\sin (2 n-1) \theta}{\sin \theta}=\left(\frac{\sin n \theta}{\sin \theta}\right)^2$$
 \end{lemma}

Omitimos la prueba aquí ya que el resultado se sigue por inducción y se puede consultar en \cite{murty2007problems}.

 \begin{theorem}
     Para todo entero $m\geq 0$,

     $$(2 m+1)+2 \sum_{j=0}^{2 m-1}(j+1) \cos (2 m-j) \theta=\left(\frac{\sin \left(m+\frac{1}{2}\right) \theta}{\sin \frac{\theta}{2}}\right)^2.$$
 \end{theorem}

 \begin{proof}
     Cambiando el orden de sumación, podemos reescribir la identidad como  sigue 

     $$2 m+1+2 \sum_{j=1}^{2 m}(2 m-j+1) \cos j \theta=\left(\frac{\sin \left(m+\frac{1}{2}\right) \theta}{\sin \frac{\theta}{2}}\right)^2 .$$

     Tomando $\theta=2\varphi$, debemos probar que

     $$2 m+1+2 \sum_{j=1}^{2 m}(2 m-j+1) \cos 2 j \varphi=\left(\frac{\sin (2 m+1) \varphi}{\sin \varphi}\right)^2,$$

     por el teorema 4.4

     $$1+2 \sum_{j=1}^n \cos 2 j \varphi=\frac{\sin (2 n+1) \varphi}{\sin \varphi}.$$

     Ahora note que 

     \begin{align*}
         \sum_{n=0}^{2m}(1+2 \sum_{j=1}^n \cos 2 j \varphi)&=(2m+1)+2 \sum_{n=0}^{2m}\sum_{j=1}^n \cos 2 j \varphi\\
         &=\sum_{n=0}^{2m}\frac{\sin (2 n+1) \varphi}{\sin \varphi}
     .\end{align*}

     De esto se sigue que,

     \begin{align*}
         (2m+1)+2 \sum_{n=0}^{2m}\sum_{j=1}^n \cos 2 j \varphi&=(2 m+1)+2 \sum_{j=1}^{2 m} \cos 2 j \varphi \sum_{j \leq n \leq 2 m} 1\\
         &=(2 m+1)+2 \sum_{j=1}^{2 m}(2 m-j+1) \cos 2 j \varphi\\
         &=\sum_{n=0}^{2 m} \frac{\sin (2 n+1) \varphi}{\sin \varphi}\\
         &=\left(\frac{\sin (2 m+1) \varphi}{\sin \varphi}\right)^2
     .\end{align*}
 \end{proof}

