%!TEX root = ../main.tex

\thispagestyle{empty}
\vspace{-0.7cm}

\cleanchapterquote{Hasta el día de hoy, los matemáticos han intentado en vano descubrir algún orden en la secuencia de números primos, y tenemos razones para creer que es un misterio al que la mente humana nunca penetrará}{Leonhard Euler}{}

Para comenzar con este capítulo presentaremos el teorema fundamental de la aritmética (TFA), una pieza crucial en cualquier trabajo sobre teoría de números.\\


\begin{theorem}[TFA]
Todo entero $n>1$ se puede escribir como producto de primos de manera única salvo el orden de los factores, es decir:

$$n=\prod_{j=1}^{m}p_j^{k_j}.$$
\end{theorem}

Escribiremos $p^m\mid\mid n$ siempre que si $p^m\mid n$ entonces $p^{m+1}\nmid n$, es decir, $p^m$ es la potencia exacta que divide a $n$, esto nos permite escribir el TFA como:

$$n= \prod_{p^m\mid\mid n}p^m.$$

\section{Funciones aritmética}

\begin{definition}
Una función aritmética es una función con dominio los naturales y codominio $\R$ o $\C$, es decir $a$ es función aritmética si

$$a:\N \to \mathbb{F},$$

con $\mathbb{F}=\C$ o $\mathbb{F}=\R$.
\end{definition}

Esta definición nos muestra que las funciones aritmética no son más que sucesiones de números reales o complejos, en algunos casos será útil considerarlas de esta  manera y de manera análoga a  las sucesiones las denotaremos como $a_n$, donde cada $a_n$ representa $f(n)$. Veamos algunos ejemplos importantes:

\begin{itemize}
\item[$\bullet$] \textbf{Función constante k:}

$$k(n)=k, \text{ para todo } n \in \N. $$

\item[$\bullet$] \textbf{Función unidad:}

$$e(n)=\begin{cases}
1 \quad n=1\\
0 \quad n\neq 1.
\end{cases}$$

\item[$\bullet$] \textbf{Función número de divisores:} $\tau(n)$, el número de divisores positivos de $n$ (incluyendo 1 y n)

$$\tau(n)=\sum_{j\mid n}1.$$

\item[$\bullet$] \textbf{Función suma de divisores:} $\sigma(n)$, la suma de los divisores positivos de $n$

$$\sigma(n)=\sum_{j\mid n}j.$$

\item[$\bullet$] \textbf{Función de Möbius:} $\mu(n)$, se define como

$$\mu(n)=\begin{cases}
1  &n=1\\
0  &\text{si n no es libre de cuadrados}\\
(-1)^k &\text{si n tiene k factores primos.}
\end{cases}$$

\item[$\bullet$] \textbf{Función phi de Euler:} $\varphi(n)$, el número de enteros positivos $m\leq n$ que son primos relativos a n ($(m,n)=1$)

$$\varphi(n)=\sum_{\substack{m=1 \\(m, n)=1}}^n 1.$$

\item[$\bullet$]\textbf{Función de Von Mangoldt:} $\Lambda(n)$, se define como

$$\Lambda(n)= \begin{cases}\log p & n=p^m \\ 0 & \text { en otro caso. }\end{cases}$$

\item[$\bullet$]\textbf{Función identidad:} $N(n)$, la función identidad se define como:

$$N(n)=n.$$ 

\end{itemize}

Por la naturaleza de $\N$, existen dos clases importantes de funciones aritmética, las funciones aditivas y multiplicativas:

\begin{itemize}

\item[$\bullet$] Las funciones aditivas que satisfacen
$$
f(m n)=f(m)+f(n) \quad \text { siempre que }(m, n)=1,
$$


\item[$\bullet$] las funciones multiplicativas que satisfacen
$$
f(m n)=f(m) f(n) \text { siempre que }(m, n)=1 .
$$


\end{itemize}


Si una función aditiva o multiplicativa satisface la  propiedad para cualquier par de números naturales $m$ y $n$, se dirá que la función es completamente aditiva o completamente multiplicativa, respectivamente, las funciones aditivas y multiplicativas están determinadas por sus valores en las potencias de los números primos.\\ 

\begin{proof}
Supongamos que $f$ es aditiva y $n>1$, por el TFA:

$$f(n)=f\left(\prod_{p^m\mid\mid n}p^m\right)=\sum_{p^m\mid \mid n} f(p^{m}).$$

Ahora, si $f$ es multiplicativa:

$$f(n)=f\left(\prod_{p^m\mid\mid n}p^m\right)=\prod_{p^m\mid\mid n}f(p^m).$$
\end{proof}

Si además la función es completamente multiplicativa:

$$f(n)=f\left(\prod_{i=1}^{m}p_i^{k_i}\right)=\prod_{i=1}^{m} f(p_i)^{k_i},$$

lo que también ocurre para funciones completamente aditivas, cambiando el producto por una suma.  Una propiedad adicional que será útil para caracterizar estas funciones es que si $f$ es aditiva y no idénticamente nula, entonces para algún $n$, $f(1\cdot  n)=f(1)+f(n)$, así $f(1)=0$, análogamente si $f$ es multiplicativa y no idénticamente nula $f(1\cdot n)=f(1)f(n)$, $f(1)=1$.\\

Ahora veamos que aunque la función de Von Mangoldt, parece extraña, su definición es natural y nos permite obtener una versión logarítmica del teorema fundamental de la aritmética.

\begin{theorem}
Dado $n\in \N$, $n>1$ entonces:

$$\log(n)=\sum_{j\mid n}\Lambda(j).$$
\end{theorem}

\begin{proof}

Note que si $n>1$, entonces por el TFA:

$$\log(n)=k_1\log(p_1)+\ldots+k_m\log(p_m),$$

 donde los $p_j$ de la igualdad son los primos de su descomposición y $k_j$ sus potencias respectivas. Así, esta igualdad nos dice que en el cálculo de $\log(n)$ solo importan los valores del $\log$ en los divisores primos o potencias de primos, luego:

 $$\log(n)=\sum_{j\mid n}\Lambda(j).$$
\end{proof}

Sin embargo, la principal motivación para introducir la función de Von Mangoldt es que sus sumas parciales $\displaystyle\sum_{n \leq x} \Lambda(n)$ son la suma ponderada de las potencias de primos $p^m \leq x$, tomando como peso $\log p$, el peso correcto para compensar la densidad de primos. No es difícil demostrar que las potencias $p^m$ con $(m \geq 2)$ contribuyen poco en la suma anterior.\\

De hecho, estudiar el comportamiento asintótico de la suma anterior resultará equivalente a estudiar el de la función de contadora de primos $\pi(x)$; aún más, el TNP es equivalente a la afirmación

$$
\lim _{x \rightarrow \infty} \frac{1}{x} \sum_{n \leq x} \Lambda(n)=1 .
$$

Esta equivalencia nos dará el camino a la prueba del teorema de los números primos, lo que la convierte en una función aritmética muy importante.\cite{hildebrand2006introduction}

\begin{definition}

Las funciones $\psi(x)$ y $\vartheta(x)$ de Chevyshev se definen como sigue:

$$\vartheta(x)=\sum_{p\leq x}\log p, \quad \quad \psi(x)=\sum_{n \leq x} \Lambda(n).$$

\end{definition}

\subsection{La función de Möbius}

Es natural preguntarse por la definición de la función de Möbius, ya que de todas parece ser la más extraña, uno se preguntaría si hay una forma de motivarla... \\

Consideremos la función: 

$$L(x)=\sum_{n\leq x}\log(n).$$

Note que aplicando el teorema anterior

\begin{equation}
L(x)=\sum_{n\leq x}\log(n)=\sum_{n\leq x}\sum_{j\mid n}\Lambda(j).
\end{equation}

Vamos a aplicar una técnica muy útil y frecuente en teoría de números, el cambio de orden de sumación, para esto vamos a cambiar $n$ y $j$ de orden en la doble suma (1.1) y conservaremos la condición $j\mid n$.

\begin{align*}
    L(x)&=\sum_{n\leq x}\log(n)=\sum_{n\leq x}\sum_{j\mid n}\Lambda(j)\\
    &=\sum_{j \leq x} \sum_{\substack{n \leq x \\
j \mid n}} \Lambda(j)\\
&=\sum_{j \leq x} \Lambda(j) \sum_{\substack{n \leq x \\
j \mid n}} 1
,\end{align*}

ahora, ¿cuántos enteros positivos $n\leq x$ hay tal que $j\mid n$?, pues exactamente $\dfrac{x}{j}$, así

\begin{align*}
    L(x)&=\sum_{j\leq x}\Lambda(j)\sum_{m\leq \frac{x}{j}}1,\\
\end{align*}
y cambiando nuevamente el orden de sumación

\begin{align*}
    L(x)&=\sum_{m\leq x}\sum_{j\leq \frac{x}{m}}\Lambda(j)\\
    &=\sum_{m\leq x}\psi\left(\frac{x}{m}\right).
\end{align*}

Esta identidad la abordaremos más adelante, pero de momento sabemos que podemos escribir a $L(x)$ en términos de $\psi(x)$, ¿y si queremos lo opuesto?, ie. a $\psi(x)$ en términos de  $L(x)$, ¿podemos \textbf{invertir} el papel de las funciones?. Vamos a abordar esta pregunta poniéndola en un contexto más general.\\

Siguiendo a \cite{levinson1969motivated}, supongamos $F(x)$ y $G(x)$ funciones aritmética con $G(x)=\displaystyle\sum_{n\leq x}F\left(\dfrac{x}{n}\right)$, tenemos que:

$$G \left( \frac{x}{2} \right)=F \left( \dfrac{x}{2} \right)+F \left( \frac{x}{4} \right)+F \left( \frac{x}{6} \right)+\ldots,$$

así:

$$G(x)-G \left( \frac{x}{2} \right)=F(x)+F \left( \dfrac{x}{3} \right)+F \left( \dfrac{x}{5} \right)+\ldots,$$

podemos pensar que continuar restando los términos $G \left( \dfrac{x}{j} \right)$ nos permitirá obtener la inversión, sin embargo el término $G \left( \dfrac{x}{3} \right)$ contiene a $F \left( \dfrac{x}{6} \right)$, por tanto

$$
G(x)-G\left(\frac{x}{2}\right)-G\left(\frac{x}{3}\right)=F(x)+F\left(\frac{x}{5}\right)-F\left(\frac{x}{6}\right)+F\left(\frac{x}{7}\right)+\ldots,
$$
así, en los siguientes pasos debemos eliminar $-F\left(\dfrac{x}{6}\right)$. Esto se lograría sumando $G\left(\dfrac{\mathrm{x}}{6}\right)$ y no restándolo. La suma anterior nos muestra además que no necesitamos restar 

$G\left(\dfrac{x}{4}\right)$ pues $F\left(\dfrac{x}{4}\right)$ ya desapareció al restar $G\left(\dfrac{x}{2}\right)$.\\


Así, podemos intuir que necesitamos multiplicar $G(\frac{x}{j})$ en cada sumando, por una función que nos de el signo adecuado (sume y reste, según se necesite) o anule el término, como ocurre en el caso de $G\left(\frac{x}{4}\right)$. Denotemos esta función que estamos buscando como $\mu(x)$. Si suponemos que existe dicha función, entonces

\begin{equation}
F(x)=\sum_{j \leqslant x} \mu(j) G\left(\frac{x}{j}\right).
\end{equation}

Además, ya tenemos algunos valores de $\mu$, $\mu(1)=1$, $\mu(2)=\mu(3)=-1, \mu(4)=0$ y $\mu(6)=1$. Podemos de momento darnos cuenta que estos valores parecen coincidir con los que obtendríamos al evaluar la función de Möbius, lo cual no es ninguna coincidencia, sin embargo aún no podemos afirmar que son en esencia la misma función. Note que por la definición de $G$

\begin{equation}
G\left(\frac{\mathrm{x}}{\mathrm{j}}\right)=\sum_{k\leq \frac{x}{j}} F\left(\frac{x}{j \mathrm{k}}\right),
\end{equation}

por tanto al reemplazar (1.3) en (1.2), obtenemos:

\begin{align*}
    F(x)&=\sum_{j \leqslant x} \mu(j)\sum_{jk\leq x} F\left(\frac{x}{j \mathrm{k}}\right)=\sum_{jk\leq x}\mu(j)F \left( \frac{x}{jk} \right)\\
    &=\sum_{n\leq x}F \left( \frac{x}{n} \right)\sum_{jk=n}\mu(j)
.\end{align*}

Finalmente

\begin{equation}
F(x)=F(x)+\sum_{1<n\leq x}F \left( \frac{x}{n} \right)\sum_{jk=n}\mu(j).
\end{equation}

Para obtener la inversión necesitamos que la doble suma en (1.4) se anule, y dado que no tenemos condiciones sobre $F$, la función $\mu$ debe cumplir que si $n \neq 1$

$$\sum_{jk= n}\mu(j)=0$$

En efecto, \textit{la función que cumple esta propiedad es... la función de Möbius}.

\begin{theorem}
Sea $n \geqslant 1$, entonces

$$\sum_{\mathrm{d} \mid \mathfrak{n}} \mu(\mathrm{d})=e(n).$$

\end{theorem}
Antes de continuar con la prueba de este resultado notemos que la suma en (1.2) en realidad no recorre los $j\leqslant x$, sino los $j$ que son divisores de $x$ ya que $G$ es función aritmética y por tanto $\frac{x}{j}$ es necesariamente un número natural. Así

\begin{equation}
F(x)=\sum_{j \mid x} \mu(j) G\left(\frac{x}{j}\right)
\end{equation}

Esta suma sobre los divisores de $n$ llevará el nombre de convolución o producto de Dirichlet y nos permitirá darle al conjunto de las funciones aritmética una estructura de monoide abeliano, estas ideas sin embargo las estudiaremos en la  siguiente sección. Ahora continuemos con la prueba.\\

\begin{proof}

Si $n=1$, entonces $1=e(1)=\mu(1)$, si $n\neq 1$, entonces por el teorema fundamental de la aritmética $n=\displaystyle\prod_{i=1}^k p_i^{\alpha_i}$, note que los únicos divisores $d$ tales que $\mu(d)\neq 0$ son los que toman la forma $d=p_{i_1}\ldots p_{i_j}$ donde $\mathcal{K}=\{i_1,\ldots, i_j\}\subseteq \{1,\ldots,k\}$, en este caso $\mu(d)=(-1)^{|\mathcal{K}|}$. Necesitamos saber cuántas veces va a aparecer este valor en la suma, es decir dado un $0\leq r\leq k$ fijo, ¿cuántos subconjuntos de $\{1,\ldots,k\}$ tienen cardinal $r$?, exactamente $\displaystyle \binom{k}{r}$. Así la suma toma la forma:

\begin{align*}
    \sum_{d\mid n}\mu(d)&=1+\sum_{i}\mu(p_i)+\sum_{i,j}\mu(p_ip_j)+\ldots+\mu(p_1\ldots p_k)\\
    &=1-k+\binom{k}{2}+\ldots+(-1)^k\\
    &=\sum_{r=1}^k\binom{k}{r}(-1)^r=(1-1)^k=0
.\end{align*}

\end{proof}

Aplicando esto a la función $L(x)$, obtenemos que

$$\psi(x)=\sum_{j\mid n} \mu(j)L \left( \frac{n}{j} \right).$$

La fórmula en (1.5) se conoce como inversión de Möbius, las ideas aquí sin embargo fueron abordadas de manera informal, para poder presentar un argumento riguroso, necesitamos, como se menciono antes, introducir la convolución de Dirichlet, que además nos permitirá obtener propiedades importantes de algunas de las funciones aritmética que hemos presentado en esta sección.

\section{Convolución de Dirichlet}

Siguiendo las ideas de la  sección anterior, presentamos la siguiente definición:
\pagebreak

\begin{definition}
Sean f y g funciones aritméticas. Definimos la convolución o producto de Dirichlet como

$$(f*g)(n)=\sum_{\mathrm{d} \mid \mathrm{n}} \mathrm{f}(\mathrm{d}) \mathrm{g}\left(\frac{\mathrm{n}}{\mathrm{d}}\right).$$

O simplemente $f*g$.
\end{definition}

Algunos resultados del capítulo anterior se pueden escribir en términos de convolución, por ejemplo, el TFA se puede presentar como

$$\log n=\sum_{j\mid n}\Lambda(j)=\Lambda * 1,$$

donde 1, denota la función constante 1, también $\psi(x)=\mu*L$, pero la convolución no solo se introduce como una manera de simplificar notación, como mencionamos antes, esta tiene propiedades importantes que nos permitirán darle una estructura algebraica a las funciones aritmética.

\begin{theorem}
Sean $f$ y $g$ funciones aritméticas. Entonces se cumple lo siguiente

\begin{itemize}

\item[$\bullet$] $f * g=g * f$.

\item[$\bullet$] $(f * g) * h=f *(g * h)$.

\item[$\bullet$] $e * f=f * e=f$.

\end{itemize}

\end{theorem}


\begin{proof}

Primero note que $\displaystyle\sum_{j\mid n} f(j)g \left( \frac{n}{j} \right)=\sum_{jk=n} f(j)g(k)$, ya que en ambos casos la suma recorre los divisores de $n$, luego

$$\begin{aligned}
(f * g)(n) & =\sum_{j_1 j_2=n} f\left(j_1\right) g\left(j_2\right )=\sum_{j_1 j_2=n} g\left(j_1\right) f\left(j_2\right) \\
& =(g * f)(n).
\end{aligned}$$

Ya que no importa el orden en el la suma recorra los divisores, lo que prueba la conmutatividad. Ahora, recordemos que $e(n)=1$ si $n=1$ y 0 si $n\neq 1$, por tanto
$$(e*f)(n)=(f*e)(n)=\sum_{j\mid n}f(j)e \left( \frac{n}{j} \right).$$

Así, como $e \left( \dfrac{n}{j} \right)=0$ si $j\neq n$, los términos de la suma son cero excepto cuando $j=n$, 

$$(e*f)(n)=(f*e)(n)=\sum_{j\mid n}f(j)e \left( \frac{n}{j} \right)=f(n)=f.$$

Para probar la asociatividad, considere $N=g*h$ y $M=f*g$, luego

$$
\begin{aligned}
(f * N)(n) & =\sum_{j \mid n} f(j) N\left(\frac{n}{j}\right) \\
& =\sum_{j_1 j_2=n} f\left(j_1\right) N\left(j_2\right) \\
& =\sum_{j_1 j_2=n} f\left(j_1\right)\left(\sum_{j_3 j_4=j_2} g\left(j_3\right) h\left(j_4\right)\right) \\
& =\sum_{j_1 j_3 j_4=n} f\left(j_1\right) g\left(j_3\right) h\left(j_4\right)\\
& =\sum_{j_1 j_3 j_4=n} f\left(j_3\right) g\left(j_4\right) h\left(j_1\right)\\
& =\sum_{j_1 j_2=n}\left(\sum_{j_3 j_4=j_2} f\left(j_3\right) g\left(j_4\right)\right) h\left(j_1\right)\\
& =\sum_{j_1 j_2=n} M\left(j_2\right) h\left(j_1\right)\\
&=(M * h)(n).
\end{aligned}
$$

\end{proof}

Hemos  probado en particular que la función $e$ es el elemento neutro de la convolución, sabemos además que $\mu *1=e$, es decir la función de Möbius tiene inverso multiplicativo, con estas nuevas herramientas podemos presentar una prueba corta y rigurosa de la fórmula de inversión de Möbius (1.5).

\begin{theorem}[Fórmula de inversión de Möbius]
Sean $f$ y $g$ funciones aritmética, entonces $f=g*1$ si y solo si $g=\mu* f$.
\end{theorem}

\begin{proof}
Note que $f=g*1$ si y solo si $\mu*f=\mu*g*1=g*\mu*1=g*e=g$
\end{proof}

Sin embargo, no toda función aritmética tiene inverso multiplicativo, el caso más evidente es tomar la función constante $N=0$, note que para toda $f$, $f*N=N$. Esto nos lleva a la pregunta: ¿bajo qué condiciones una función aritmética tiene inverso multiplicativo?, la respuesta podría venir de estudiar las características que no permiten que $N$ lo tenga... A saber, $N$ \textit{se anula en todo punto}, ¿bastaría con que esta función no se anule en todo su dominio para que tenga inversa?, o ¿en algún punto en particular?, la respuesta nos viene del siguiente teorema, basta con que la función no se anule en 1 para poder garantizar además la \textit{unicidad}.

\begin{theorem}
 
Sea $f$ una función aritmética tal que $f(1) \neq 0$. Entonces existe una única función aritmética $g$ tal que $f*g=e$.

\end{theorem} 

\begin{proof}
Note que si $n=1$, entonces $f(1)g(1)=e(1)=1$, así $g(1)=\dfrac{1}{f(1)}$, ahora  supongamos que $g$ se ha definido para todos lo valores $1<k<n$, luego como $f*g(n)=0$:

\begin{equation}
0=\sum_{d\mid n}g(d)f \left( \frac{n}{d} \right)=\sum_{\substack{d \mid n \\
d<n}} f\left(\frac{n}{d}\right) g(d)+f(1) g(n),
\end{equation}

así

$$g(n) =\frac{-1}{f(1)} \sum_{\substack{d \mid n \\
d<n}} f\left(\frac{n}{d}\right) g(d).1$$

Esto nos define $g$ de manera recursiva, lo que concluye el resultado.

\end{proof}


Esto nos permite dotar a estas funciones aritmética de una estructura de grupo Abeliano, ya que si $f(1)\neq 0$ y $g(1)\neq 0$, entonces $f*g(1)=f(1)g(1)\neq 0$.\\


\begin{theorem}
Sean $f$ y $g$ funciones aritméticas multiplicativas, entonces $f * g$ también es multiplicativa.
\end{theorem}

\begin{proof}
Sean $x, y \in \mathbb{N}$ tal que $(x, y)=1$. Note que cada divisor $d \mid xy$ puede escribirse de manera única como $d=mn$ donde $m \mid x$ y $n \mid y$, además $(m, n)=1$ y $\left(\dfrac{x}{m}, \dfrac{y}{n}\right)=1$, por lo tanto

$$
\begin{aligned}
(f * g)(xy) & =\sum_{d \mid xy} f(d) g\left(\frac{xy}{d}\right) \\
& =\sum_{\substack{m|x \\
n| y}} f(mn) g\left(\frac{xy}{mn}\right) \\
& =\sum_{\substack{m|x \\
n| y}} f(m) g\left(\frac{x}{m}\right) f(n) g\left(\frac{y}{n}\right) \\
& =\sum_{m \mid x} f(m) g\left(\frac{y}{m}\right) \sum_{n \mid y} f(n) g\left(\frac{y}{n}\right) \\
& =(f * g)(x)(f * g)(y) .
\end{aligned}
$$

Así $f * g$ es multiplicativa.
\end{proof}


\begin{theorem}
Si $f$ es multiplicativa, entonces $g=f^{-1}$ también es multiplicativa.
\end{theorem}

Donde $f^{-1}$ denota su inversa, presentaremos una prueba siguiendo a \cite{hildebrand2006introduction}\\

\begin{proof}
Queremos ver que:

\begin{equation}
g(n_1n_2)=g(n_1)g(n_2) \quad \text{si} \quad (n_1,n_2)=1.
\end{equation}

Procedamos por inducción matemática. Sea $n=n_1 n_2$, si $n_1 n_2=1$, entonces $n_1=n_2=1$, luego

$$g(1\cdot 1)=g(1)=\dfrac{1}{f(1)}=1=g(1)g(1).$$

Supongamos ahora que $g$ satisface (1.7) para todo $k_1k_2\geq 2$ tal que $k_1k_2<n$ y sean $n_1$ y $n_2$ tales que $n_1n_2=n$ y $(n_1,n_2)=1$, por (1.6) tenemos que:

$$
\begin{aligned}
0 & =\sum_{d \mid n_1 n_2} f(d) g\left(\frac{n_1n_2}{d}\right) \\
&=\sum_{\substack{d_1 \mid n_1 \\
d_2\mid n_2\\ d_1d_2<n}} f(d_1)f(d_2) g \left( \frac{n_1}{d_1} \right)g \left( \frac{n_2}{d_2} \right)+ g(n_1n_2)\\
&=\sum_{\substack{d_1 \mid n_1 \\
d_2\mid n_2}} f(d_1)f(d_2) g \left( \frac{n_1}{d_1} \right)g \left( \frac{n_2}{d_2} \right)+ g(n_1n_2)-g(n_1)g(n_2)\\
& =(f * g)\left(n_1\right)(f * g)\left(n_2\right)+\left(g\left(n_1 n_2\right)-g\left(n_1\right) g\left(n_2\right)\right),
\end{aligned}
$$

luego:

$$g\left(n_1\right) g\left(n_2\right)=e\left(n_1\right) e\left(n_2\right)+g\left(n_1 n_2\right),$$

\vspace*{0.2cm}

y como $n_1n_2\geq 2$, entonces $n_1\geq 2$ o $n_2\geq 2$, así $e(n_1)e(n_2)=0$, por tanto $g(n_1n_2)=g(n_1)g(n_2)$.
\end{proof}

\begin{corollary}
Sea $\mathcal{M}$ el conjunto de funciones aritmética multiplicativas, entonces $(\mathcal{M},*)$ es un grupo Abeliano.
\end{corollary}

\subsection{Propiedades de algunas funciones aritmética}

Finalizaremos esta sección con algunas propiedades importantes de las funciones aritmética que definimos el inicio del capítulo.

\vspace*{-0.5cm}

\subsubsection{La función \texorpdfstring{$\varphi$}{Lg} de Euler}

La propiedad del teorema (1.3) nos permite manipular sumas  con condiciones de coprimalidad, es decir sumas sobre los $n$ que son coprimos con un entero $k$ fijo. Considere el conjunto $C_k=\{n \text{ | } (n,k)=1\}$, note que la función característica del conjunto $C_k$ es:

$$
\mathbbm{1}_{C_k}(n)=\sum_{d \mid(n, k)} \mu(d)=e((n,k)),
$$

Una aplicación de esto, nos permite obtener la siguiente propiedad de la función $\varphi$ de Euler:

$$\begin{aligned}
\varphi(n) & =\sum_{\substack{m \leq n \\
(m, n)=1}} 1=\sum_{m\leq n}\mathbbm{1}_{C_n}(m) \\
& =\sum_{m \leq n} \sum_{d \mid(m, n)} \mu(d) \\
& =\sum_{d \mid n} \mu(d) \sum_{\substack{m \leq n \\
d \mid m}} 1 \\
& =\sum_{d \mid n} \mu(d) \frac{n}{d}\\
&=\mu*N(n)=n \sum_{d \mid n} \frac{\mu(d)}{d} .
\end{aligned}$$

Es claro que la función $N$ es multiplicativa por definición, luego esta propiedad nos permite probar que la función $\varphi$ es multiplicativa, por el teorema (1.7) basta ver que en efecto $\mu$ lo es.



\begin{theorem}
La función $\mu$ es multiplicativa:
\end{theorem}

\begin{proof}

Supongamos que $n=n_1n_2$, si $n=1$, entonces $n_1=n_2=1$, $\mu(n)=\mu(n_1)\mu(n_2)=1$. Ahora supongamos que $\mu(k)=\mu(k_1)\mu(k_2)$, para todo $k=k_1k_2$ tal que $1<k<n$ y $(k_1,k_2)=1$ y sean $n_1,n_2$ tales que $n_1n_2=n$ y $(n_1,n_2)=1$, tenemos que

\begin{align*}
    0&=\sum_{d\mid n_1n_2}\mu(d)\\
    &=\sum_{\substack{d_1\mid n_1\\ d_2\mid n_2\\d_1d_2<n}}\mu(d_1)\mu(d_2)+\mu(n_1n_2)\\
    &=\sum_{\substack{d_1\mid n_1}}\mu(d_1)\sum_{d_2\mid n_2}\mu(d_2)+\mu(n_1n_2)-\mu(n_1)\mu(n_2)\\
    &=\mu(n_1n_2)-\mu(n_1)\mu(n_2)
.\end{align*}

Así, por el principio de inducción matemática se sigue el resultado.
\end{proof}

\begin{corollary}
La función $\varphi(n)$ tiene las siguientes propiedades:

\begin{itemize}
\item[i)] $\varphi(mn)=\varphi(m)\varphi(n)$ si $(m,n)=1$

\item[ii)] $\varphi(p^n)=p^n-p^{n-1}$

\item[iii)] $\varphi(n)=n\displaystyle\prod_{p\mid n}\left(1-\frac{1}{p}\right)  $
\end{itemize}
\end{corollary}

\begin{proof}

i) Como $\varphi=\mu* N$, se sigue del teorema anterior.\\

ii) Note que si $n=p^k$, entonces

\begin{align*}
    \varphi(p^k)&=p^k\sum_{j\mid p^k}\frac{\mu(j)}{j}\\
    &=p^k\left(1+\frac{\mu(p)}{p}+\frac{\mu(p^2)}{p^2}+\ldots+\frac{\mu(p^k)}{p^k}\right)\\
    &=p^k \left(1-\frac{1}{p}\right)=p^k-p^{k-1}
.\end{align*}

iii) Sea $n>1$, por el TFA se sigue que

\begin{align*}
    \varphi(n)&= \varphi \left( \prod_{p^m\mid\mid n}p^m \right)\\
    &= \prod_{p^m\mid\mid n}\varphi(p^m)\\
    &=  \prod_{p^m\mid\mid n} p^m-p^{m-1}\\
    &=n \prod_{p\mid n}\left( 1-\frac{1}{p} \right)
.\end{align*}
\end{proof}

Al inicio del capítulo mencionamos que las funciones aritmética están totalmente determinadas por sus valores en las potencias de primos, esta propiedad nos permite probar de manera sencilla afirmaciones del estilo $f*g=h$, siempre que $f,g$ y $h$ sean funciones multiplicativas, basta ver que $f*g(p^m)=h(p^m)$, veamos un ejemplo:

\begin{theorem}
La función $\varphi(n)$ satisface la propiedad:

$$n=\sum_{j\mid n}\varphi(j)$$
\end{theorem}

\begin{proof}
La afirmación se puede  escribir como $N=\varphi*1$, como estas funciones son multiplicativas, entonces basta ver que la identidad se tiene en las potencias de primos, en efecto

\begin{align*}
    \sum_{j\mid p^m}\varphi(j)&=\varphi(1)+\varphi(p)+\varphi(p^2)+\varphi(p^3)+\ldots+\varphi(p^m)\\
    &=1+(\slashed{p}-1)+(\slashed{p^2}-\slashed{p})+(\slashed{p^3}-\slashed{p^2})+\slashed{ } \ldots\not{ } +(p^m-\slashed p^{m-1})\\
    &=p^m
.\end{align*}
\end{proof}

Esta identidad  también puede probarse usando propiedades  de la convolución

\begin{align*}
    \sum_{j\mid n}\varphi(j)=\varphi*1(n)=(N*\mu)*1(n)=N*(\mu*1)(n)=N*e(n)=n.
\end{align*}

\subsubsection{Las funciones número y suma de divisores}

También  podemos aplicar la convolución para obtener también propiedades de la funciones suma y número de divisores, por ejemplo:

\begin{equation}
\sigma(n)=\sum_{j\mid n}j=N*1(n) \quad \text{y} \quad \tau(n)=\sum_{j\mid n}1=1*1(n).
\end{equation}

Note que $\sigma*\varphi=(N*1)*(\mu*N)=(N*e)*N=N*N$, en efecto:

$$\sigma*\varphi(n)=N*N(n)=\sum_{j\mid n}j\cdot\frac{n}{j}=n\sum_{j\mid n}1=n\tau(n).$$

Con lo que obtenemos una propiedad  interesante que relaciona estas 3 funciones aritmética, pero  además (1.8) nos dice también que las funciones $\sigma$ y $\tau$ son multiplicativas por ser convolución de funciones multiplicativas. Observemos una última propiedad de estas funciones, que nos permite caracterizar la noción de primalidad.

\begin{prop}
Un entero $n$ es primo si y solo si $\sigma(n)+\varphi(n)=n\tau(n)$.
\end{prop}

\begin{proof}

Si $n$ es primo, entonces $\varphi(n)=n-1$, $\sigma(n)=n+1$ y $\tau(n)=2$, luego es claro que $\sigma(n)+\varphi(n)=n\tau(n)$. Veamos ahora que si $n$ no es primo entonces no se sigue el teorema\\

Si $n$ no es primo entonces $\varphi(n)<n-1$, ahora note que

$$\sigma(n)=\sum_{j\mid n}j=1+\sum_{\substack{j\mid n\\ j>1}}j\leq 1+n(\tau(n)-1).$$

Luego:

$$\sigma(n)+\varphi(n)<n-1+1+n\tau(n)-n=n\tau(n).$$
\end{proof}

\section{Sumación Parcial}

Los resultados que hemos podido obtener hasta el momento solo se centran en casos finitos, en sumas sobre los divisores de un entero $n$ fijo o sobre los $k$ que son primos relativos a $n$, estos son casos privilegiados, nuestro objetivo sigue siendo el TNP, para esto necesitamos poder obtener relaciones asintóticas, algunas funciones como $\mu$ o $\varphi$ aparentan comportamientos caóticos al graficarlas en función de $n$ y por tanto no tiene mucho sentido estudiar un comportamiento asintótico para  ellas, sin embargo, algunas funciones aritmética $f(n)=a_n$ tienen buen comportamiento en la media, en el sentido de que sus sumas parciales:

$$A(x)=\sum_{n\leq x}a_n.$$

Tienden a ser suaves  conforme $x\to \infty$ y frecuentemente podemos estudiarlas de manera precisa, ejemplo de esto son $\pi(x)$ o $\psi(x)$. En esta sección estudiaremos algunos de los métodos principales para obtener dichas estimaciones, estimaciones que nos darán un camino a la prueba del TNP.\\

\begin{definition}
    Sea $f$ una función real definida en $[a,b]$. Suponga que $f(x+)$ y $f(x-)$ existen para todo $x \in (a,b)$. Definimos
       \begin{itemize}
           \item $f(x)-f(x-)$: Salto a izquierda de $f$ en $x$.
           \item $f(x+)-f(x)$: Salto a derecha de $f$ en $x$.
           \item $[f(x)-f(x-)]+[f(x+)-f(x)]=f(x+)-f(x-)$: Salto de $f$ en $x$.
       \end{itemize}
\end{definition}

\begin{note}
Diremos que $f(x)=O(g(x))$ cuando existen constantes $M>0$ y $x_0$ tales que para todo $x>x_0$, se cumple que $|f(x)| \leq M|g(x)|$, es decir, nuestro dominio específico es $x>x_0$.\\
\end{note}

Una \textbf{fórmula asintótica} para $f(x)$ es una expresión de la forma $f(x) \sim g(x)$, mientras que una \textbf{estimación asintótica} para $f(x)$ es una expresión del tipo $f(x)=$ $g(x)+O(R(x))$, donde $g(x)$ representa el término principal y $R(x)$ el término de error. La notación $O$ grande nos permite entonces controlar el error de estimaciones asintóticas de manera precisa, cosa que no ocurre con $o$ pequeña, recordemos que:

$$f(x)=o(g(x)): \quad \displaystyle \lim_{x \to \infty} \frac{f(x)}{g(x)}=0.$$

Esto intuitivamente nos dice que $f(x)$ ``crece más lento'' que $g(x)$, pero esto no nos da tanta información acerca de $f$ con O grande, ya que no sabemos la ``velocidad'' con la que este cociente tiende a 0.\\

Las estimaciones con O también son más fáciles de trabajar y manipular que las estimaciones con o. Por ejemplo, las O se pueden ``sacar'' de integrales o sumas, siempre que las funciones involucradas sean no negativas, mientras que tales manipulaciones generalmente no se permiten con las o.\\

\subsection{La integral de Riemann-Stieltjes}

Sean $P=\{x_0,x_1,...,x_n\}\in\mathcal{P}[a,b]$ (las particiones del intervalo $[a,b]$) y   $t_{k}\in [x_{k-1},x_k]$ cualesquiera. Una suma de la forma $$S(P,f)=\sum_{k=1}^{n}f(t_k)\Delta x_k,$$ se denomina \textbf{suma de Riemann} de $f$ en $[a,b]$, recordamos que $\Delta x_k=x_k-x_{k-1}$.\\

La integral usual se define como el límite de las sumas de Riemann, esta definición nos da una correspondencia importante: Toda integral de Riemann se puede ver como el límite de una suma, una serie.\\

Sin embargo, no toda suma se puede ver como una integral bajo esta definición, si tenemos una con una condición de sumación sobre, por ejemplo, los números primos, no hay un camino claro para expresarla como una integral, la solución a este problema viene de generalizar la noción de integral.\\

La propiedad deseada la obtendremos de la integral de Riemann-Stieltjes.

\begin{definition}
\phantom{uwu}
\begin{itemize}

    \item[i)] Sean $P=\{x_0,x_1,...,x_n\}\in\mathcal{P}[a,b]$ (las particiones del intervalo $[a,b]$) y   $t_{k}\in [x_{k-1},x_k]$ cualesquiera. 
    
    Una suma de la forma $$S(P,f,\alpha)=\sum_{k=1}^{n}f(t_k)\Delta \alpha_k,$$ se denomina \textbf{suma de Riemann-Stieltjes} de $f$ con respecto a $\alpha$ en $[a,b]$.
    
    \item[ii)] Decimos que $f$ es \textbf{Riemann-Integrable} con respecto a $\alpha$ en $[a,b]$, y escribimos ``$f\in \mathcal{R}(\alpha)$ en $[a,b]$'', si existe $A\in \mathbb R$ que satisface que\\ 
    Para todo $\varepsilon>0$ existe $P_{\varepsilon}\in \mathcal{P}[a,b]$ tal que si para toda $P\supset P_{\varepsilon}$ y para cualquier elección de puntos $t_{k}\in [x_{k-1},x_k]$, entonces $$\mid S(P,f,\alpha) - A\mid <\varepsilon.$$
    \end{itemize}
\end{definition}

Donde $\Delta\alpha_k=\alpha(x_k)-\alpha(x_{k-1})$, cuando $A$ existe, es único y se denota por $$\int_a^b fd\alpha \quad \text{o} \quad \int_{a}^{b} f(x)d\alpha(x).$$ 
La función $f$ es llamada \textit{integrando} y la función $\alpha$ es llamada \textit{integrador}.\\


Note que la integral de Riemann no es más que un caso particular de la de Riemann-Stieltjes, cuando $\alpha(x)=x$, esta integral y sus propiedades se suelen estudiar en un curso de Análisis II, aquí recordaremos algunas de  las más importantes.\\

\begin{theorem}[\cite{Apostol:105425}, Teorema 7.11]
  Sea $\alpha$ una función escalonada definida en $[a, b]$ con salto $\alpha_k$ en $x_k$.\\
  
Sea $f$ una función definida en $[a, b]$ tal que $f$ y $\alpha$ no sean ambas discontinuas a la derecha o a la izquierda de cada $x_k$. Entonces $\displaystyle\int_a^b f d \alpha$ existe y se tiene que:
$$
\int_a^b f(x) d \alpha(x)=\sum_{k=1}^n f\left(x_k\right) \alpha_k.
$$
\end{theorem}

La prueba de esto se encuentra en \cite{Apostol:105425}, sin embargo, nos será útil explorar la idea.\\

Note que $\alpha$ es constante en los intervalos $(x_{k-1},x_k)$, luego por intervalos la integral es 0 ya que para cualquier suma de Riemann-Stieltjes, $S(P,f,\alpha)$=0. Así, para conocer el valor de la integral entonces solo tendríamos que sumar el valor que toma alrededor de cada $x_k$.\\

Supongamos que $\alpha$ tiene salto $\alpha_k $ en un punto $x_k\in [a,b]$, no es difícil ver que:

\begin{equation}
\int_a^bfd\alpha=f(x_k)[\alpha(x_k+)-\alpha(x_{k}-)]=f(x_k)\alpha_k.
\end{equation}

La prueba de esto se encuentra también en \cite{Apostol:105425} [Teorema 7.9], en efecto si repetimos esto para cada $x_k$ obtenemos la suma deseada.\\

Sea $f$ continua en $[0,N]$, el teorema anterior nos permite expresar la suma $\displaystyle\sum_{n=1}^Na_nf(n)$ como una integral de Riemann Stieltjes considerando sus sumas parciales

$$\displaystyle\sum_{n=1}^Na_nf(n)=\int_0^Nf(x)dA(x).$$

Ya que al aplicar (1.9) tomando como integrador las sumas parciales tenemos que $A(n+)-A(n-)=A(n)-A(n-1)=a_n$, luego en cada paso la integral toma el valor $a_nf(n)$.\\

\begin{note}
El límite inferior de la integral puede ser cualquier número en el intervalo [0, 1) y el superior cualquier número en [N, N+1)
\end{note}

\begin{eg}
\phantom{uwu}
\begin{itemize}
\item Si se toma $\alpha(x)=[x]$, entonces
$$
\int_a^b f(x) d[x]=\sum_{a<n \leq b} f(n).
$$

\item Si se toma $\alpha(x)=\pi(x)$, la función contadora de primos, que tiene un salto de 1 en cada p primo, entonces
$$
\int_a^b f(x) d \pi(x)=\sum_{a<p \leq b} f(p).
$$
\end{itemize}
\end{eg}

\subsection{Algunas propiedades de la integral de Riemann-Stieltjes}


\begin{theorem}[Integración por partes]\label{integrac partes}
    Si $f\in \mathcal{R}(\alpha)$ en $[a,b]$ entonces $\alpha \in\mathcal{R}(f)$ en $[a,b]$ y $$\int_{a}^{b}f(x)d\alpha(x)+\int_{a}^{b}\alpha(x)df(x)=f(b)\alpha(b)-f(a)\alpha(a).$$
\end{theorem}

Bajo ciertas condiciones una integral de Riemann-Stieltjes se puede reducir a una integral de Riemann usual, esto es muy útil puesto que estamos más familiarizados con el cálculo de estas, dichas condiciones son presentadas en el siguiente teorema:

\begin{theorem}\label{reduccion a riemann}
    Sea $ f \in R(\alpha) $ en $[a, b]$, donde $ \alpha \in C^1[a, b]$, entonces,
$ \displaystyle\int_{a}^{b} f(x)\alpha'(x)dx$
existe y

\[ \int_{a}^{b} f \,d\alpha = \int_{a}^{b} f(x)\alpha'(x)dx. \]
\end{theorem}

Las pruebas de estos teoremas se encuentran también en \cite{Apostol:105425}, por lo que no las presentaremos aquí para no extendernos demasiado en la teoría de esta sección, con estas propiedades ya podemos presentar una prueba  del teorema de  sumación de Abel.

\begin{theorem}[Sumación parcial de Abel]\label{Teorema sumacion de Abel}
Sea $a_n$ función aritmética y $f\in C^1[1,x]$, entonces:

$$
\sum_{n \leq x} a_n f(n)=A(x) f(x)-\int_1^x A(t) f^{\prime}(t) d t .
$$
\end{theorem}

\begin{proof}
Tenemos que:

$$\sum_{n \leq x} a_n f(n)=\int_0^xf(t)dA(t)=f(x)A(x)-\int_0^xA(t)df(t),$$

ya que $A(x)=0$ para todo $x\in [0,1)$, en efecto

\begin{align*}
    \sum_{n \leq x} a_n f(n)&=f(x)A(x)-\int_1^xA(t)df(t)\\
    &=f(x)A(x)-\int_1^xA(t)f^{'}(t)dt
.\end{align*}

Por  el teorema \ref{reduccion a riemann}.
\end{proof}

\section{Algunas estimaciones básicas}

Vamos a ver algunas aplicaciones de la teoría que hemos presentado, algunas de estas ya las habíamos mencionado antes, por ejemplo, el comportamiento asintótico de las sumas de la serie armónica.\\

Vamos a ver que $\displaystyle\sum_{n \leq x} \frac{1}{n} =\log x+\gamma+O\left(\frac{1}{x}\right)$, donde $\gamma=\displaystyle\lim_{n \to \infty} \left( \sum_{n=1}^{n}\frac{1}{n} -\log n\right)$ conocida como la constante de Euler-Mascheroni.\\

Sea $a_n=1$ y $f(x)=\dfrac{1}{x}$, así $A(x)=[x]$ y al usar la sumación parcial tenemos que


$$
\begin{aligned}
\sum_{n \leq x} \frac{1}{n} & =\frac{[x]}{x}+\int_1^x \frac{[t]}{t^2} d t \\
& =\frac{x-\{x\}}{x}+\int_1^x \frac{t-\{t\}}{t^2} d t.
\end{aligned}
$$

Donde $\{x\}$ denota la parte fraccionaria de $x$, es decir $\{x\}=x-[x]$, note que $\{x\}=O(1)$, luego

\begin{align*}
   \sum_{n \leq x} \frac{1}{n}&=1+O\left(\frac{1}{x}\right)+\log x-\int_1^x \frac{\{t\}}{t^2} d t\\
   &=1+O\left(\frac{1}{x}\right)+\log x-\left(\int_1^{\infty} \frac{\{t\}}{t^2} d t-\int_x^{\infty} \frac{\{t\}}{t^2} d t\right)
.\end{align*}

Basta ver que 1-$\displaystyle\int_1^{\infty}\frac{\{x\}}{x^2}dx=\gamma$ ya que la otra integral también es del orden de $O\left(\dfrac{1}{x}\right)$, en efecto

$$
\begin{aligned}
\gamma & =\lim _{n \rightarrow \infty}\left(\sum_{k=1}^n \frac{1}{k}-\log n\right) \\
& =\lim _{n \rightarrow \infty}\left(1+\sum_{1<k \leq n} \frac{1}{k}-\log n\right) \\
& =1+\lim _{n \rightarrow \infty}\left(\int_1^n \frac{1}{x} d[x]-\int_1^n \frac{1}{x} d x\right) \\
& =1+\lim _{n \rightarrow \infty}\left(\int_1^n \frac{[x]}{x^2} d x-\int_1^n \frac{1}{x} d x\right)\\
&=1-\int_1^{\infty}\dfrac{\{x\}}{x^2}dx.
\end{aligned}
$$

Esto concluye el resultado que mencionamos en la introducción, $H(n)\thicksim \log n$. Veamos otro ejemplo:

\begin{eg}
Estimación de las sumas parciales de $\log n$:\\

Sea $a_n=1$ y $f(x)=\log x$, así $A(x)=[x]$, luego:
$$
\begin{aligned}
\sum_{n \leq x} \log n & =[x] \log x-\int_1^x \frac{[t]}{t} d t \\
& =(x-O(1)) \log x-\int_1^x \frac{t-O(1)}{t} d t \\
& =x \log x-O(\log x)-(x-1)+O(\log x) \\
& =x \log x-x+O(\log x) .
\end{aligned}
$$
\end{eg}

\subsection{Una equivalencia importante}

Vamos a obtener finalmente que el TNP es equivalente a la afirmación $\psi(x)\thicksim x$

\begin{definition}
Sea $x\in \mathbb{N}$, con $x>1$, definimos la función contadora de primos $\pi(x)$ como: 

    $$\pi(x)=\displaystyle\sum_{p\leq x}1.$$
\end{definition}


 Como vimos antes, la fórmula de sumación  de Abel tiene un gran poder teórico que explotaremos en distintos lugares de este trabajo. Por lo pronto ella será esencial para para estimar $\vartheta(x)$ y $\pi(x)$, de donde obtendremos la equivalencia deseada.

\begin{theorem}\label{rep integral de pi x}
Tenemos las siguientes identidades:
    $$
    \begin{aligned}
    & \vartheta(x)=\pi(x) \log x-\int_2^x \frac{\pi(t)}{t} d t, \\
    & \pi(x)=\frac{\vartheta(x)}{\log x}+\int_2^x \frac{\vartheta(t)}{t \log ^2 t} d t .
    \end{aligned}
    $$
\end{theorem}

\begin{proof}
Sea  $\mathbbm{1}_{p}$ la función característica del conjunto de primos. Tenemos entonces las siguientes igualdades:

$$\vartheta(x)=\sum_{n \leqslant x} \mathbbm{1}_p(n) \log (n)\quad \text{y}\quad \pi(x)=\sum_{n \leqslant x} \mathbbm{1}_p(n)$$

Fijemos $x \geqslant 2$. Por la fórmula de sumación de Abel y como $\pi(t)=0$ para todo $t<2$, tenemos que:
$$
\begin{aligned}
\vartheta(x) & =\pi(x) \log (x)-\int_1^x \frac{\pi(t)}{t} d t \\
& =\pi(x) \log (x)-\int_2^x \frac{\pi(t)}{t} d t
\end{aligned}
$$

Notando que $\pi(x)=\displaystyle\sum_{n \leqslant x} \dfrac{\mathbbm{1}_p(n) \log (n)}{\log (n)}$ y que $\vartheta(t)=0$ para todo $t<2$ tenemos que:

$$
\begin{aligned}
\pi(x) & =\frac{\vartheta(x)}{\log (x)}+\int_1^x \frac{\vartheta(t)}{t \log ^2(t)} d t \\
& =\frac{\vartheta(x)}{\log (x)}+\int_2^x \frac{\vartheta(t)}{t \log ^2(t)} d t
\end{aligned}
$$
\end{proof}

Ahora vamos a establecer una conexión entre las dos funciones de Chebyshev que definimos antes, en efecto:

$$
\psi(x)=\sum_{n \leq x} \Lambda(n)=\sum_{m=1}^{\infty} \sum_{p^m \leq x} \Lambda\left(p^m\right)=\sum_{m=1}^{\infty} \sum_{p \leq x^{\frac{1}{m}}} \log p .
$$

Note que la suma sobre $m$ realmente es finita porque la suma sobre $p$, se  detiene cuando $x^{\frac{1}{m}}<2$, es decir, cuando $m>\log_2(x)$, entonces:

\begin{align}
    \psi(x)=\sum_{m \leq \log _2 x} \sum_{p \leq x^{\frac{1}{m}}} \log p=\sum_{m \leq \log _2 x} \vartheta\left(x^{\frac{1}{m}}\right)
\end{align}

\begin{theorem}
    Si $x>0$ se tiene que:
$$
0 \leq \frac{\psi(x)}{x}-\frac{\vartheta(x)}{x} \leq \frac{\log ^2 x}{\sqrt{x} \log 4} .
$$
\end{theorem}


\begin{proof}

Por (1.10) se sigue que:

    $$
\psi(x)=\sum_{2 \leq m \leq \log _2 x} \vartheta\left(x^{\frac{1}{m}}\right)+\vartheta(x) .
$$

Ya que si $m=1$ entonces $\vartheta(x^{\frac{1}{m}})=\vartheta(x)$, luego:

$$
\psi(x)-\vartheta(x)=\sum_{2 \leq m \leq \log _2 x} \vartheta\left(x^{\frac{1}{m}}\right)\geq  0.
$$

Ahora por definición de $\vartheta:$

$$
\vartheta(x)=\sum_{p \leq x} \log p\leq x \log x .
$$

Entonces

\begin{align*}
    0 \leq \psi(x)-\vartheta(x) &\leq \sum_{2 \leq m \leq \log _2 x} x^{\frac{1}{m}} \log x^{\frac{1}{m}} \\
    &\leq \sqrt{x} \sum_{2 \leq m \leq \log _2 x} \log x^{\frac{1}{m}}\\
    &\leq \sqrt{x}(\log_2(x)\log(\sqrt{x}))\\
    &=\frac{\sqrt{x}\log^2 x}{\log(4)}
\end{align*}
\end{proof}

\begin{theorem}
    La afirmación $\psi(x)\thicksim x$ es equivalente a $\vartheta(x)\thicksim x$
\end{theorem}

\begin{proof}
    Por el teorema anterior:

    $$0\leq \dfrac{\psi(x)}{x}-\dfrac{\vartheta(x)}{x}\leq \dfrac{\log^2(x)}{\sqrt{x}\log(4)}$$

    y como $\lim_{x\to \infty} \dfrac{\log^2(x)}{\sqrt{x}\log(4)}=0$, entonces cuando $x\to \infty$ se tiene que $\dfrac{\psi(x)}{x}-\dfrac{\vartheta(x)}{x}=0$, lo que concluye el resultado.
\end{proof}


\begin{theorem}
\label{equivalencia tnp}
    Las siguientes afirmaciones son equivalentes:
    \begin{itemize}
        \item[i)] $\pi(x)\thicksim \dfrac{x}{\log(x)}$

        \item [ii)] $\vartheta(x)\thicksim x$

        \item[iii)] $\psi(x)\thicksim x$
    \end{itemize}    
\end{theorem}

Por el teorema anterior basta ver que $i \leftrightarrow ii$\\

 \begin{proof} Por el teorema \ref{rep integral de pi x} y dado que estamos trabajando con aproximaciones asintóticas, podemos asumir que $x\geq 2$, se sigue que:

 \begin{align}
     \frac{\vartheta(x)}{x}=\frac{\pi(x) \log (x)}{x}-\frac{1}{x} \int_2^x \frac{\pi(t)}{t} d t
 \end{align}
 \begin{align}
     \frac{\pi(x) \log (x)}{x}=\frac{\vartheta(x)}{x}+\frac{\log (x)}{x} \int_2^x \frac{\vartheta(t)}{t \log ^2(t)} d t
 \end{align}

Basta con ver que las integrales de (1.11) y (1.12) van a 0 , cuando $x \rightarrow \infty$.\\

$(\rightarrow)$ Por hipótesis $\dfrac{\pi(x)}{x}\left(\dfrac{1}{\log (x)}\right)^{-1}=1$ cuando $x \rightarrow \infty$. Esto es equivalente a decir que $\dfrac{\pi(t)}{t}=O\left(\dfrac{1}{\log (t)}\right)$. Luego para todo $x \geq 2$ positivo fijo tenemos que:
$$
\frac{1}{x} \int_2^{\mathrm{x}} \frac{\pi(\mathrm{t})}{\mathrm{t}} \mathrm{dt}=O\left(\frac{1}{\mathrm{x}} \int_2^{\mathrm{x}} \frac{1}{\log (\mathrm{t})} \mathrm{dt}\right)
$$

Note que:
$$
\begin{aligned}
\int_2^x \frac{1}{\log (t)} d t & \leq \int_2^{\sqrt{x}} \frac{1}{\log (t)} d t+\int_{\sqrt{x}}^x \frac{1}{\log (t)} d t \\
& \leq \frac{\sqrt{x}}{\log (2)}+\frac{x-\sqrt{x}}{\log (\sqrt{x})}
\end{aligned}
$$

Luego
$$
\frac{1}{x} \int_2^x \frac{1}{\log (t)} d t \leq \frac{1}{\sqrt{x} \log (2)}+\frac{1}{\log (\sqrt{x})}-\frac{1}{\sqrt{x} \log (\sqrt{x})}
$$

Así, cuando $x \rightarrow \infty, \dfrac{1}{x} \displaystyle \int_2^x \frac{1}{\log (t)} d t=0$.\\

$(\leftarrow)$ Análogamente $\vartheta(\mathrm{t})=$ $O(\mathrm{t})$. Por tanto:

$$
\frac{\log (x)}{x} \int_2^x \frac{\vartheta(t)}{t \log ^2(t)} d t=O\left(\frac{\log (x)}{x} \int_2^x \frac{1}{t \log ^2(t)} d t\right)
$$
 
 La integral en $O$ se puede acotar de manera análoga a la anterior:
 
$$
\begin{aligned}
\int_2^x \frac{1}{t \log ^2(t)} d t & =\int_2^{\sqrt{x}} \frac{1}{t \log ^2(t)} d t+\int_{\sqrt{x}}^x \frac{1}{t \log ^2(t)} d t \\
& \leq \frac{\sqrt{x}}{\log ^2(2)}+\frac{x-\sqrt{x}}{\log ^2(\sqrt{x})}
\end{aligned}
$$

Multiplicando ambos lados por $\dfrac{\log (x)}{x}$, podemos ver que si $x \rightarrow \infty$, $\dfrac{\log (x)}{x}\displaystyle \int_2^x \frac{1}{t \log ^2(t)}=0$
\end{proof}

Finalizaremos con una aplicación interesante de la fórmula de sumación de Abel:

\begin{prop}[Fórmula de Stirling]
Si $n$ es un entero positivo, entonces:

$$n!=C \sqrt{n} n^n e^{-n}\left(1+O\left(\frac{1}{n}\right)\right)$$
\end{prop}

\begin{proof}
Por el teorema \ref{Teorema sumacion de Abel}:

\begin{align*}
    \sum_{k\leq n}\log k&=n\log n-n+1+\int_1^n\frac{\{t\}}{t}dt\\
    &=n\log n -n+1+\dfrac{1}{2}\log n+\int_1^n \frac{s(t)}{t}dt
\end{align*}

Donde $s(t)=\{t\}-\frac{1}{2}$, note que aplicando integración por partes:

\begin{align*}
\int_1^n \frac{s(t)}{t}=\left.\frac{S(t)}{t}\right|_1 ^n+\int_1^n \frac{S(t)}{t^2} dt
\end{align*}

Donde $S(t)=\displaystyle\int_1^t s(y)dy$, tenemos que $\displaystyle\left.\frac{S(t)}{t}\right|_1 ^n=0$ ya que $s(t)$ es una función periódica de periodo 1 y $\displaystyle\int_k^{k+1}s(t)dt=0$ para todo entero $k$. Note que $|S(t)|\leq \dfrac{1}{2}$, de esto se sigue que:

\begin{align*}
\int_1^n \frac{s(t)}{t}dt&=\int_1^n\frac{S(t)}{t^2}=\int_1^{\infty}\frac{S(t)}{t^2}dt-\int_n^{\infty}\frac{S(t)}{t^2}dt\\
&=c+O\left(\frac{1}{x}\right) && \left(|S(t)|\leq \frac{1}{2}\right)
\end{align*}

Por tanto:

$$\sum_{k\leq n}\log k=n \log n-n+\frac{1}{2} \log n+c+O\left(\frac{1}{n}\right)$$

De esto se sigue que:

\begin{align*}
    n!&=exp{\displaystyle\left(\sum_{k\leq n}\log k\right)}\\
    &=exp{\displaystyle \left(n \log n-n+\frac{1}{2} \log n+c+O\left(\frac{1}{n}\right)\right)}\\
    &=C\sqrt{n}n^ne^{-n}\left(1+O\left(\frac{1}{n}\right)\right)
.\end{align*}

\end{proof}

Se puede ver que $C=\sqrt{2\pi}$, sin embargo no abordaremos eso en este trabajo.

\section{Series de Dirichlet}

Dada una función aritmética $f$, se le pueden asignar a esta dos tipos de series importantes:

$$
\begin{aligned}
& E(z)=\sum_{n=1}^{\infty} f(n) z^n, \\
& F(s)=\sum_{n=1}^{\infty} \frac{f(n)}{n^s},
\end{aligned}
$$
estas son llamadas funciones generatrices de $f$.
Las series de potencia generatrices fueron introducidas por Euler con el propósito de estudiar problemas de naturaleza aditiva, esto ya que tienen la siguiente propiedad:\\

$$
E_f(z) E_g(z)=\sum_{n=1}^{\infty} h(n) z^n,
$$
donde
$$
h(n)=\sum_{a+b=n} f(a) g(b) .
$$

Note que $h(n)$ se parece la convolución de Dirichlet, pero en una versión aditiva, por eso estas series toman un papel importante para este tipo de problemas.\\

Sin embargo, nuestro interés son las series $F(s)$, estas fueron introducidas por Dirichlet en su trabajo sobre primos en progresiones aritmética y son particularmente útiles cuando $f$ es una función multiplicativa ya que entre muchas cosas, en el producto de dos de  estas aparece la convolución multiplicativa que vimos antes, por lo que a través de estas podemos obtener propiedades de $f(n)$ como en la sección 1.2.\\

Con lo que se mencionó antes parece que toda la teoría se conecta, y que de cierta forma esto solo es una forma diferente de atacar los mismos problemas, sin embargo, este no es el caso, las propiedades analíticas de una serie de Dirichlet, vista como función de variable compleja $s$ pueden ser explotadas para obtener información importante acerca del comportamiento de las sumas parciales de funciones aritmética:

$$\sum_{n\leq x}f(n)$$

\begin{definition}

Sea $f$ una función aritmética, la serie:

$$F(s)=\sum_{n=1}^{\infty}\frac{f(n)}{n^s}$$

Es llamada \textit{serie de Dirichlet asociada a} f

\end{definition}

La variable $s$ es usualmente escrita como $s=\sigma+it$ con $\sigma=\Re(s)$ y $t=\Im(s)$, la serie de Dirichlet más popular es la función zeta de Riemann $\zeta(s)$, definida como la serie de Dirichlet asociada a la constante 1:

$$\zeta(s)=\sum_{n=1}^{\infty} \frac{1}{n^s}, \quad (\sigma>1) $$

\subsection{Propiedades algebraicas}

\begin{theorem}\label{proddirich}
Sean $f$ y $g$ las funciones aritmética asociadas a $F(s)$ y $G(s)$. Sea $h=f*g$ la convolución de $f$ y $g$ y $H(s)$ su serie de Dirichlet asociada. Si $F(s)$ y $G(s)$ convergen absolutamente en algún punto $s$, entonces también $H(s)$ y $H(s)=F(s)G(s)$. 
\end{theorem}

\begin{proof}
Note que por la convergencia absoluta:
$$
\begin{aligned}
F(s) G(s) &=\sum_{k=1}^{\infty} \frac{f(k)}{k^s}\sum_{m=1}^{\infty} \frac{g(m)}{m^s} \\
&=\sum_{k=1}^{\infty} \sum_{m=1}^{\infty} \frac{f(k) g(m)}{k^s m^s} \\
& =\sum_{n=1}^{\infty} \frac{1}{n^s} \sum_{k m=n} f(k) g(m)=\sum_{n=1}^{\infty} \frac{f * g(n)}{n^s},
\end{aligned}
$$

Además:

$$
\begin{aligned}
\sum_{n=1}^{\infty}\left|\frac{h(n)}{n^s}\right| & \leq \sum_{n=1}^{\infty} \frac{1}{\left|n^s\right|} \sum_{k m=n}|f(k)||g(m)| \\
& =\left(\sum_{k=1}^{\infty}\left|\frac{f(k)}{k^s}\right|\right)\left(\sum_{m=1}^{\infty}\left|\frac{g(m)}{m^s}\right|\right)\\
&<\infty
\end{aligned}
$$
\end{proof}

Es muy importante tener en cuenta que esto no se puede garantizar sin la convergencia absoluta, ya que los reordenamientos que realizamos en la prueba no son posibles.

\begin{corollary}

Sean $f$ una función aritmética con serie de Dirichlet asociada $F(s)$, $g$ tal que $f*g=e$ y $G(s)$ la serie de Dirichlet asociada a $g$, entonces $G(s)=\dfrac{1}{F(s)}$ en cualquier punto $s$ en el que $F(s)$ y $G(s)$ sean ambas absolutamente convergentes.

\end{corollary}
\pagebreak

\begin{proof}
Note que la serie de Dirichlet asociada a $e$ es 1, luego por el teorema anterior:

$$1=\sum_{n=1}^{\infty} \frac{e(n)}{n^s}=\sum_{n=1}^{\infty} \frac{f*g(n)}{n^s}=F(s)G(s)$$
\end{proof}

\begin{note}
La convergencia absoluta de \( F(s) \) no implica la de la serie de Dirichlet asociada con su inversa. Por ejemplo, la función definida por \( f(1)=1 \), \( f(2)=-1 \), y \( f(n)=0 \) para \( n \geq 3 \) tiene la serie de Dirichlet \( F(s)=1-2^{-s} \), que converge para todo $s$. Sin embargo, la serie de Dirichlet de la inversa de Dirichlet de \( f \) es \( \dfrac{1}{F(s)} = (1-2^{-s})^{-1} = \displaystyle\sum_{k=0}^{\infty} \frac{1}{2^{ks}} \), que converge absolutamente en \( \sigma > 0 \), pero no en el semiplano \( \sigma \leq 0 \), ya que es una serie geométrica. \cite{hildebrand2006introduction}\\
\end{note}

Si tomamos una serie de potencias $F(x)$, sabemos por lo menos tres cosas de ella:

\begin{itemize}[label=$\bullet$]
    \item $F$ tiene un disco de convergencia, 

    \item En el interior del disco, $F$ converge absolutamente.

    \item En el interior del disco, $F$ es una función analítica \cite{Apostol:105425}
\end{itemize}

Nuestro objetivo ahora es lograr obtener propiedades similares para series de Dirichlet.
\subsection{Propiedades analíticas}

Primero note que $\displaystyle x^{\displaystyle  s}=e^{\displaystyle s\log x}=e^{\displaystyle(\sigma+it)\log x}=x^{\displaystyle\sigma}e^{\displaystyle it\log{x}}$, por tanto $|\displaystyle x^{\displaystyle s}|=|x^{\displaystyle\sigma}|$ ya que:
\begin{align*}
     \left|e^{ i\theta}\right|&=|cos(\theta)+i\sin(\theta)|\\
     &=1
 .\end{align*} 

luego $\left| e^{\displaystyle it\log x}\right|=1$. Esto muestra que la convergencia absoluta de una serie de Dirichlet está determinada únicamente por $\sigma$.

\begin{eg}[Convergencia de $\zeta(s)$]
Note que:

$$\sum_{n=1}^{\infty} \left|\frac{1}{n^s}\right|=\sum_{n=1}^{\infty} \left|\frac{1}{n^{\sigma}}\right|=\sum_{n=1}^{\infty} \frac{1}{n^{\sigma}}$$

Si $\sigma=1$ sabemos que la serie diverge, en otro caso, la serie converge si y solo si la integral:

$$\int_1^{\infty}\frac{1}{x^\sigma}dx=\lim_{n \to \infty}\left.\dfrac{x^{1-\sigma}}{1-\sigma}\right|_1^{n}=\lim _{n \rightarrow \infty} \frac{n^{1-\sigma}}{1-\sigma}-\frac{1}{1-\sigma}$$

Luego, es claro que $\zeta(s)$ converge absolutamente si $\sigma>1$ y diverge  si $\sigma\leq 1$
\end{eg}

\begin{definition}
Sea $\Omega$ un abierto en $\C$ y $f:\Omega \to \C$, la \textit{derivada} de $f$ en $z_0\in \Omega$ está dada por:

$$f^{\prime}(z_0)=\lim_{z \to z_0} \frac{f(z)-f(z_0)}{z-z_0}$$

Si el límite existe. Diremos que $f$ es \textit{diferenciable} en $z_0$
\end{definition}

Si $f$ es diferenciable en todo punto de una vecindad de $z_0$, entonces diremos que $f$ es \textit{holomorfa} en $z_0$. Si $f$ es holomorfa en todo punto de un abierto $U\subseteq\Omega$, diremos que $f$ es holomorfa en $U$. Finalmente si $f$ es holomorfa en $\Omega$, diremos que es holomorfa.\\

Por otro lado, $f$ se dice analítica en $z_0$, si existe un abierto $U\subseteq \Omega$ tal que $z_0\in U$ y

$$f(z)=\sum_{n=0}^{\infty} a_n (z-z_0)^n$$

para todo $z\in U$, si $f$ es analítica en todo punto de $U$, entonces decimos que es analítica  en $U$.\\

El conjunto de puntos $s=\sigma+it$ con $\sigma>a$ es llamado semiplano, veremos que toda serie de Dirichlet tiene un semiplano $\sigma>\sigma_c$ en el que la serie converge y otro semiplano $\sigma>\sigma_a$ en el que la serie converge absolutamente. Las constantes $\sigma_c$ y $\sigma_a$ son llamas abscisa de convergencia y abscisa de convergencia absoluta respectivamente. Si la serie de Dirichlet converge para todo $s\in \C$, decimos que $\sigma_c=-\infty$, si no converge para todo $s\in \C$, decimos que $\sigma_c=\infty$, de manera análoga lo haremos con la convergencia absoluta.\cite{pongsriiam2023analytic}


\begin{theorem}
Sea $f:\N \to \C$, supongamos que la serie:

$$F(s)=\sum_{n=1}^{\infty} \left| \frac{f(n)}{n^s}\right|$$

No converge, ni diverge para  todo $s\in \C$, entonces existe un $\sigma_a\in \R$ tal que $F(s)$ converge para todo $\sigma>\sigma_a$ y diverge para todo $\sigma<\sigma_a$
\end{theorem}

\begin{proof}
Sea $|n^s|=n^{\sigma}$, si $\sigma>\sigma_0$, entonces $|n^s|\geq n^{\sigma_0}$, así $|f(n)n^{-s}|\leq|f(n)|n^{-\sigma_0}$. Por el criterio de comparación, si $F(s)$ converge para $s=\sigma_0+i t_0$, entonces converge para todo $s$ con parte real $\sigma\geq\sigma_0$. Considere el conjunto:

$$A:\left\{a\in R : \sum_{n=1}^{\infty} \left|\frac{f(n)}{n^a}\right|<\infty\right\}$$

Como $F(s)$ no diverge para  todo $s\in\C $, entonces $A$ es no vacío, además tiene un mínimo ya que $F(s)$ no converge  para todo $s\in C$, sea $\sigma_a$ el mínimo de $A$. Si $s\in C$ tiene parte real $\sigma< \sigma_a$, entonces $\sigma\not\in A$, $F(s)$ diverge. Análogamente si $s$ tiene parte real $\sigma>\sigma_a$, entonces $\sigma>a$ para algún $a\in A$, $F(s)$ converge.\\
\end{proof}


\begin{definition}
Decimos que una serie de funciones $\{f_n\}:S\to \C$ converge uniformemente a $f$ en $S$ si, para todo $\varepsilon>0$, existe un $N_\varepsilon$ tal que si $n>N_\epsilon$ implica que
    $$|f_n(x)-f(x)|<\varepsilon\ \ \ \ \text{para cada $x$ in $S$}$$
\end{definition}

Lo importante a resaltar en la definición anterior es  que nos indica que para cualquier $\varepsilon$, podemos encontrar un $N_\varepsilon$ tal de ese punto en adelante, la sucesión de funciones está muy cerca de $f$ para todo $x\in S$, es decir que ese $N_\varepsilon$ no depende del punto $x$, es uniforme, cosa que no ocurre con la convergencia puntual.

\begin{theorem}
Sean $\Omega$ un abierto en $\C$, $(f_n)$ una sucesión de funciones analíticas definidas en $\Omega$, y $f:\Omega\to\C$. Si $(f_n)$ converge absolutamente a $f$ en cualquier subconjunto compacto de $\Omega$, entonces $f$ es analítica en $\Omega$ y la sucesión $(f_n^{\prime})$ también converge uniformemente a $f^{\prime}$ en cualquier subconjunto compacto de $\Omega$.
\end{theorem}

No entraremos en los detalles de esta prueba aquí, el lector interesado la puede  consultar en \cite{stein2010complex}.

\begin{theorem}
Suponga que la serie $F(s)=\displaystyle\sum_{n=1}^{\infty} \frac{f(n)}{n^s}$ converge en el punto $s=s_0$ y sea $H>0$, entonces $F(s)$ converge uniformemente en $S$, donde $S$ es el conjunto:

$$S=\{s\in \C : \sigma>\sigma_0 \text{ y } |t-t_0|\leq H(\sigma-\sigma_0)\}$$
\end{theorem}
\begin{center}
\input{Graphics/Dirichlet1}
\end{center}

\begin{proof}
Sean $H>0$, $\varepsilon>0$ dado y $1<M<N$, como la serie converge en $s=s_0$, entonces:

$$\sum_{M<n \leq N} f(n) n^{-s_0} n^{s_0-s}=\int_M^N t^{s_0-s} d A(t)=\int_M^N t^{s_0-s} d(A(t)-A(M))$$

Con $A(t)=\displaystyle\sum_{n \leq t} f(n) n^{-s_0}$, luego:
$$
\begin{aligned}
\sum_{M<n \leq N} f(n) n^{-s} & =\frac{A(N)-A(M)}{N^{s-s_0}}- \int_M^N(A(t)-A(M)) d(t^{s_0-s})\\
&=\frac{A(N)-A(M)}{N^{s-s_0}}-(s_0-s)\int_M^N(A(t)-A(M))t^{s_0-s-1}dt
\end{aligned}
$$

Tenemos que $A(t)$ converge cuando $t\to\infty$, luego $A(N)-A(M)$ y $A(t)-A(M)$ convergen a 0 cuando $M,N\to \infty$. Así, para $M, N$ suficientemente grandes y $\sigma>\sigma_0$:

$$|A(N)-A(M)|<\frac{\varepsilon}{(H+2)}, \quad|A(t)-A(M)|<\frac{\varepsilon}{H+2}$$

De esto se sigue que:

$$\begin{aligned}
\left|\sum_{M<n \leq N} \frac{f(n)}{n^s}\right|&<\frac{\varepsilon}{H+2}+\frac{\varepsilon}{H+2}\left|s-s_0\right| \int_M^{\infty} t^{\sigma_0-\sigma-1} d t\\
&=\frac{\varepsilon}{H+2}+\frac{\varepsilon\left|s-s_0\right|}{(H+2)\left(\sigma-\sigma_0\right) M^{\sigma-\sigma_0}}\\
&\leq\left(1+\frac{\left|s-s_0\right|}{\sigma-\sigma_0}\right) \frac{\varepsilon}{H+2} .
\end{aligned}$$

Queremos ver que $\left|\displaystyle\sum_{M<n \leq N} f(n) n^{-s}\right|<\varepsilon$, así el teorema se sigue de el teorema de Cauchy para convergencia uniforme de series, En efecto:

$$\left|s-s_0\right|=\left|\left(\sigma-\sigma_0\right)+i\left(t-t_0\right)\right| \leq \sigma-\sigma_0+\left|t-t_0\right| \leq\sigma-\sigma_0+H(\sigma-\sigma_0)\leq(H+1)\left(\sigma-\sigma_0\right)$$

Así:

$$\left|\sum_{M<n \leq N} \frac{f(n)}{n^s}\right|<(1+(H+1))\frac{\varepsilon}{H+2}=\epsilon$$
\end{proof}


Note que haciendo $H$ suficientemente grande obtenemos que $F(s)$ converge para todo  $s$ en el semiplano $\sigma>\sigma_0$, geométricamente lo que hacemos es abrir el cono de la región $S$ para obtener este semiplano.\cite{montgomery2007multiplicative}

\begin{theorem}
Supongamos que $F(s)=\displaystyle\sum_{n=1}^{\infty} f(n)n^{-s}$ no converge para todo $s\in \C$ ni diverge para todo $s\in\C$, entonces existe $\sigma_c\in \R$ tal que $F(s)$ converge para todo $s$ con $\sigma>\sigma_0$ y diverge para todo $\sigma<\sigma_0$.
\end{theorem}


\begin{proof}
Como mencionamos antes, tomando $H$ suficientemente grande vemos que si $F(s)$ converge en $s=s_0$, entonces por el teorema anterior $F(s)$ converge para todo $\sigma>\sigma_0$, por tanto, de manera análoga a la prueba del teorema 1.25, consideremos el conjunto:

$$A=\left\{a \in \mathbb{R} : \sum_{n=1}^{\infty} f(n) n^{-a} \text { converges }\right\}$$

Nuevamente  consideremos $\sigma_c$ el mínimo de $A$, si $s$ tiene parte real $\sigma>\sigma_c$, entonces $\sigma>a$ para algún $a\in A$, luego $F(s)$ converge, si $s$ tiene parte real $\sigma<\sigma_c$, entonces $\sigma<\dfrac{\sigma+\sigma_c}{2}<\sigma_c$ y como $\sigma_c$ es el mínimo, $\dfrac{\sigma+\sigma_c}{2}\not\in A$, lo que implica que $F(s)$ diverge.
\end{proof}

\begin{eg}
Supongamos que la suma:

$$\sum_{n\leq x}f(n) \text{ está acotada para  todo }x\in [1,\infty)$$

Tenemos que $n^{-s}$ para $\sigma>0$ es decreciente, luego dado $\sigma_0>0$, tenemos que $\displaystyle\sum_{n=1}^{\infty}f(n)n^{-\sigma_0}$ converge por el criterio de Dirichlet y por el teorema anterior $F(s)$ converge para todo $S$ con $\sigma>\sigma_0$, haciendo $\sigma_0$ arbitrariamente pequeño, obtenemos que $F(s)$  converge en el semiplano $\sigma>0$
\end{eg}

El siguiente resultado nos da una propiedad que esperábamos, podemos derivar término a término:

\begin{corollary}
Sea $F(s)=\displaystyle\sum_{n=1}^{\infty} f(n)n^{-s}$ para todo $s\in \C$ con $\sigma>\sigma_c$, entonces $F(s)$ es analítica en el semiplano $\sigma>\sigma_c$ y $F^{\prime}(s)$ se puede obtener derivando término a término, es decir:

$$F^{\prime}(s)=-\sum_{n=1}^{\infty}\frac{f(n)\log n}{n^s} \quad \text{para todo } s \text{ con } \sigma>\sigma_c$$
\end{corollary}


\begin{proof}
Sea $A$ un compacto en el semiplano $\sigma>\sigma_c$, tomemos $s_0$ con parte real $\sigma_0$ tal que $\sigma_c<\sigma_0<\min\{\Re(s): s\in A\}$, como $F(s)$ converge en $s_0$, entonces por el Teorema 1.27, $F(s)$ converge uniformemente en la región $S=\{s\in \C : \sigma>\sigma_0 \text{ y } |t-t_0|\leq H(\sigma-\sigma_0)\}$. Sabemos que $A$ es acotado porque es compacto, luego existe un $H>0$ tal que $A\subset S$, basta tomar un $H$ suficientemente grande. De  esto se sigue que $F(s)$ converge uniformemente en cualquier compacto $A$ del semiplano $\sigma>\sigma_c$, por el Teorema 1.26, $F(s)$ es analítica en el semiplano $\sigma>\sigma_0$, $F^{\prime}(s)$ se puede obtener derivando término a término:

$$F^{\prime}(s)=-\sum_{n=1}^{\infty} f(n)(\log n) n^{-s}$$
\begin{center}
\input{Graphics/Analítica1}
\end{center}
\end{proof}

\begin{note}
Por lo anterior, $F^{\prime}(s)$ tiene la misma abscisa de convergencia que $F(s)$, note que el corolario anterior también se tiene si consideramos $s$ en el semiplano de convergencia absoluta, la prueba es exactamente la misma, resaltamos esto porque además nos dice que $F^{\prime}(s)$ tiene la misma abscisa de convergencia absoluta que $F(s)$, esto nos será útil después.\cite{apostol1998introduction}
\end{note}

\begin{corollary}
La función zeta de Riemann es analítica en el semiplano $\sigma>1$. Luego:

$$\zeta^{\prime}(s)=-\sum_{n=1}^{\infty}\frac{\log n}{n^s} \text{ para todo } s \in \mathbb{C} \text{ con } \sigma>1$$
\end{corollary}

Recordemos que el TFA en términos de convolución nos dice que $\log n=\Lambda*1$, por tanto:

$$\zeta^{\prime}(s)=-\sum_{n=1}^{\infty}\frac{\log n}{n^s}=-\sum_{n=1}^{\infty}\frac{\Lambda *1(n)}{n^s}=-\sum_{n=1}^{\infty}\frac{\Lambda(n)}{n^s}\sum_{k=1}^{\infty} \frac{1}{k^s} $$

El paso anterior, en el que separamos la convolución como un producto depende de que ambas series sean absolutamente convergentes en el punto $s$. No tenemos de momento una abscisa de convergencia absoluta  para la serie de Dirichlet asociada a la función de Von Mangolth, Afortunadamente, no es difícil ver que también $\sigma_a=1$. Note que para todo $n\in \N$, $|\Lambda(n)|\leq \log n$ y como la serie de Dirichlet de $\log(n)$ es $-\zeta^{\prime}(s)$ que tiene abscisa de convergencia absoluta $\sigma_a=1$, entonces:

$$\sum_{n=1}^{\infty} \frac{\Lambda(n)}{n^s}\quad \text{converge absolutamente si } \sigma>1$$

Luego el paso que hicimos es válido en el semiplano $\sigma>1$, por tanto:

$$\frac{\zeta^{\prime}(s)}{\zeta(s)}=-\sum_{n=1}^{\infty} \frac{\Lambda(n)}{n^s}$$

Esta es la derivada logarítmica, una serie de Dirichlet que será muy importante en la prueba del teorema de los números primos. El siguiente teorema nos da una relación importante entre $\sigma_c$ y $\sigma_a$.

\begin{theorem}
Sea $F(s)$ una serie de Dirichlet, entonces $\sigma_c\leq\sigma_a\leq \sigma_c+1$
\end{theorem}

\begin{proof}
Es claro que si $F(s)$ es absolutamente convergente, entonces $\sigma_a\geq \sigma_c$, veamos la otra desigualdad. Note que la serie $\displaystyle \sum_{n=1}^{\infty} f(n)n^{-\sigma_c-\epsilon}$ converge para todo $\epsilon>0$, por tanto $f(n)n^{-\sigma_c-\epsilon}$ converge a 0 cuando $n\to \infty$. Es decir, existe un $N>0$ tal que $|f(n)|<n^{\sigma_c+\epsilon}$ para todo $n\geq N$, luego:

$$\left|\frac{f(n)}{n^{\sigma+1+2\epsilon}}\right|<\frac{1}{n^{1+\epsilon}}$$

Tenemos que la serie $\displaystyle \sum_{n=1}^{\infty} n^{-1-\epsilon}$ converge, luego por criterio de comparación, la serie:

$$\sum_{n=1}^{\infty} \frac{f(n)}{n^{\sigma_c+2\epsilon+1}} \quad\text{converge absolutamente}$$

Esto es, $\sigma_a\leq \sigma_c+1+2\epsilon$, para todo $\epsilon>0$, luego obtenemos $\sigma_a\leq \sigma_c+1$
\end{proof}

\begin{theorem}[Unicidad]

Sean $F(s)$ y $G(s)$ las series de Dirichlet asociadas a $f$ y $g$ respectivamente, si $F(s)$ y $G(s)$ tienen abscisa de convergencia finita y además $F(s)=G(s)$ para  todo $s$ con $\sigma$ suficientemente grande. Entonces $f(n)=g(n)$.
\end{theorem}

\begin{proof}
Sea $h(n)=f(n)-g(n)$ y $H(s)=F(s)-G(s)$ su serie de Dirichlet, por hipótesis, existe un $\sigma_0$ tal que $H(s)$ converge en el semiplano $\sigma>\sigma_0$ y $H(s)$ es idénticamente nula en este semiplano, queremos ver que $h(n)=n$ para todo $n\in \N$. Supongamos que $h(n)$ no es idénticamente nula y sea $N$ el mínimo entero positivo tal que $h(n)\neq 0$, entonces:

\begin{align*}
    &H(s)=\frac{h(N)}{N^s}+\sum_{n=N+1}^{\infty} \frac{h(n)}{n^s},\\
    &h(N)=N^s H(s)-N^s \sum_{n=N+1}^{\infty} h(n) n^{-s}
.\end{align*}

De esto se sigue que para todo $\sigma>\sigma_0$, $h(N)=-N^{\sigma}\displaystyle \sum_{n=N+1}^{\infty} h(n) n^{-\sigma}.$ y por tanto:

$$\left|h(N)\right|\leq\sum_{n=N+1}^{\infty} |h(n)|\left(\frac{N}{n}\right)^{\sigma}
    $$

Tomando $\sigma=\sigma_0+\lambda$ con $\lambda\geq 0$, tenemos que para todo $n\geq N+1$:

$$\left(\frac{N}{n}\right)^{\sigma}=\left(\frac{N}{n}\right)^{\lambda}\left(\frac{N}{n}\right)^{\sigma_0}\leq \left(\frac{N}{N+1}\right)^{\lambda}\left(\frac{N}{n}\right)^{\sigma_0}$$

Luego:

$$\begin{aligned}
\left|h(N)\right|&\leq \left(\frac{N}{N+1}\right)^{\lambda}\sum_{n=N+1}^{\infty} |h(n)|\left(\frac{N}{n}\right)^{\sigma_0}=N^{\sigma_0}\left(\frac{N}{N+1}\right)^{\lambda}\sum_{n=N+1}^{\infty} |h(n)|n^{-\sigma_0}\\
&=C\left(\frac{N}{N+1}\right)^{\lambda}
\end{aligned}
    $$

Donde $C$ es una constante que no depende de $\lambda$ dado que la serie $\displaystyle \sum_{n=N+1}^{\infty} |h(n)|n^{-\sigma_0}$ converge. Note que:

$$\lim_{\lambda\to \infty} C\left(\frac{N}{N+1}\right)^{\lambda}=0$$

Luego $h(N)=0$, contradicción. $f(n)=g(n)$ para todo $n\in \N$.\\
\end{proof}

Con estas propiedades nos basta para lo que queremos mostrar en este trabajo, sin embargo hay muchas otras propiedades analíticas de las series de Dirichlet que son útiles en otros contextos y además muy interesantes, algunas fuentes en las que se pueden consultar son \cite{apostol1998introduction} y \cite{montgomery2007multiplicative}.

\subsection{El producto de Euler}

Ya habíamos mencionado antes que las series de Dirichlet son particularmente útiles cuando $f$ es una función multiplicativa, ya que en este caso vemos a ver que:

$$
F(s)=\prod_p\left(1+\frac{f(p)}{p^s}+\frac{f\left(p^2\right)}{p^{2 s}}++\frac{f\left(p^3\right)}{p^{3 s}}+\cdots\right),
$$

siempre que las series sobre los primos y el producto converjan absolutamente. Adicionalmente, si $f$ es completamente multiplicativa:

$$F(s)=\prod_p(1-f(p))^{-1}.$$

Este resultado es llamado producto de Euler, curiosamente no porque Euler lo haya demostrado, sino porque el primer resultado de este estilo fue dado por Euler, para la función $\zeta(s)$, $s>1$. Como sabemos, este resultado es muy importante para  poder probar la divergencia de la serie de los recíprocos de los números primos, por lo que requerimos la versión general para  poder presentar una prueba del teorema de Dirichlet en nuestro siguiente capítulo.\\


Para este punto debemos asumir algunos resultados de convergencia de productos infinitos, resultados que son bien conocidos, sin embargo, vamos a recordarlos por completitud. Dicho esto no presentaremos pruebas, estos resultados se estudian bien a detalle en \cite{Apostol:105425}, allí se encuentran pruebas detalladas y el lector interesado puede consultarlas.

\begin{definition}
Decimos que el producto $\displaystyle \prod_{n=1}^{\infty} a_n$ converge si:
\begin{itemize}[label=$\bullet$]
\item Existe $k\in \N$ tal que $a_n\neq 0$ para todo $n\geq k$ y
\item $\displaystyle\lim_{m \to \infty} \prod_{n=k}^{m}a_n$ existe y es  no nulo.
\end{itemize}
\end{definition}

De la definición anterior tenemos que un producto $\displaystyle\prod_{n=1}^{\infty} a_n$ converge a 0 si y solo si existe un $n$ tal que $a_n=0$ y hay solo una cantidad finita de $a_n$ de esta forma, así tomando $N=\max\{n: a_n=0\}$ el $\lim_{m \to \infty} \displaystyle\prod_{k=N+1}^{m}a_N$ existe y es no nulo. Por ejemplo, el producto:

$$(0)(1)(1)(1)(1)\cdots \quad \text{converge a 0.}$$

Pero los productos $(0)(1)(0)(1)(0)(1)\cdots$ y $(0)(1)(2)(3)(4)\cdots$ no convergen \cite{pongsriiam2023analytic}. De manera análoga a lo que ocurre con series, si el producto $\displaystyle\prod_{n=1}^{\infty} a_n$ converge, entonces $a_n$ converge a 1. Nos interesa caracterizar la convergencia de los productos de la forma $\displaystyle \prod_{n=1}^{\infty} (1+a_n)$, note que por lo anterior $a_n$ converge a 0.

\begin{definition}
Sea $a_n$ una sucesión de números complejos. El producto $\displaystyle\prod_{n=1}^{\infty} (1+a_n)$ es absolutamente convergente si $\displaystyle\prod_{n=1}^{\infty} (1+\left|a_n\right|)$ es convergente.
\end{definition}

En productos infinitos también la convergencia absoluta implica convergencia.

\begin{theorem}
Si la serie $\displaystyle\sum_{n=1}^{\infty} |a_n|$ converge, entonces el producto $\displaystyle\prod_{n=1}^{\infty} (1+|a_n|)$ converge.
\end{theorem}

Como mencionamos, no se presentará una prueba de este resultado, el lector puede consultarla en \cite{Apostol:105425}.

\begin{theorem}
Sea $f$ una función aritmética multiplicativa si $\displaystyle\sum_{n=1}^{\infty} f(n)$ converge absolutamente, entonces
$$\sum_{n=1}^{\infty} f(n)=\prod_p\left(1+f(p)+f\left(p^2\right)+\cdots\right)$$

Si $f$ es completamente multiplicativa, entonces:

$$\sum_{n=1}^{\infty} f(n)=\prod_p(1-f(p))^{-1}$$
\end{theorem}

\begin{proof}
Note que:

\begin{equation}
\sum_{p}\left|f(p)+f(p^2)+\ldots\right|\leq \sum_p \left|f(p)\right|+\sum_p \left|f(p^2)\right|+\ldots\leq\sum_{n=2}^{\infty} \left|f(n)\right|
\end{equation}

La serie en la derecha de (1.13) converge, entonces $\displaystyle\prod_p\left(1+f(p)+f\left(p^2\right)+\cdots\right)$ converge absolutamente. Como $f$ es multiplicativa tenemos que:

\begin{equation}
\begin{aligned}
P(x)&=\prod_{p\leq x}\left(1+f(p)+f(p^2)+\ldots\right)=\prod_{p\leq x}\left(1+\sum_{k=1}^{\infty} f(p^k)\right)\\
&=1+\sum_{k_1=1}^{\infty}f(p_1^{k_1})+\sum_{k_2=1}^{\infty}f(p_2^{k_2})+\sum_{k_1=1}^{\infty} f(p_1^{k_1})\sum_{k_2=1}^{\infty} f(p_2^{k_2})+\ldots\\
&=\sum_{k_1=0}^{\infty}\ldots \sum_{k_t=0}^{\infty}(f(p_1^{k_1})\ldots f(p_t^{k_t}))\\
&=\sum_{k_1=0}^{\infty}\ldots \sum_{k_t=0}^{\infty}(f(p_1^{k_1}\ldots p_t^{k_t}))
\end{aligned}
\end{equation}

En efecto podemos obtener el término 1 tomando $k_1=k_2=\ldots=k_t=0$ ya que $f(1)=1$ porque $f$ es multiplicativa. Note ahora que dado el conjunto $A$ de los enteros positivos con todos sus factores primos menores o iguales a $x$, el producto se puede escribir de manera compacta como:

$$P(x)=\sum_{n\in A}f(n)$$

En efecto, dado $B$ el conjunto de los enteros positivos con al menos un factor primo mayor que $x$:

$$
\left|\sum_{n=1}^{\infty} f(n)-P(x)\right| \leq \sum_{n \in B}|f(n)| \leq \sum_{n>x}|f(n)|=\sum_{n=1}^{\infty}|f(n)|-\sum_{n \leq x}|f(n)|
$$

Lo que converge a $0$ cuando $x\to \infty$, luego $P(x)$ converge a $\displaystyle \sum_{n=1}^{\infty} f(n)$ cuando $x\to \infty$. Si $f$ es completamente  multiplicativa, $f(p^k)=f(p)^k$, aplicando esto a (1.14) obtenemos lo deseado:

$$P(x)=\prod_p\left(1+f(p)+(f(p))^2+(f(p))^3+\cdots\right)=\prod(1-f(p))^{-1}$$

Ya que las series $\displaystyle\sum_{k=0}^{\infty}f(p_t^{k_t})$ en (1.14) se vuelven geométricas, cada una convergiendo a $(1-f(p_t))^{-1}$, $\left|f(p_t)\right|<1
    $, luego:

$$\sum_{n=1}^{\infty} f(n)=\prod_{n=1}^{\infty} (1-f(p))^{-1}$$
\end{proof}

\begin{corollary}[Producto de Euler]
Sea $f$ una función aritmética y $F(s)$ su serie de Dirichlet, supongamos que $F(s)$ converge absolutamente para $\sigma>\sigma_a$. Si $f$ es  multiplicativa:

$$
\sum_{n=1}^{\infty} f(n) n^{-s}=\prod_p\left(1+f(p) p^{-s}+f\left(p^2\right) p^{-2 s}+\cdots\right)
$$

Si $f$ es completamente multiplicativa:

$$
\sum_{n=1}^{\infty} f(n) n^{-s}=\prod_p\left(1-f(p) p^{-s}\right)^{-1}
$$

Y cada producto es absolutamente convergente para $\sigma>\sigma_a$
\end{corollary}

La prueba es esencialmente trivial, basta considerar:

$$F(s)=\sum_{n=1}^{\infty} f_s(n)=\sum_{n=1}^{\infty} \frac{f(n)}{n^s}$$

Y aplicar el teorema anterior a $f_s$, note que si $f$ es multiplicativa o completamente multiplicativa, también $f_s$.

\begin{corollary}
Sea $s\in \C$ tal que $\sigma>1$, entonces:

$$\zeta(s)=\prod_{p}\left(1-\frac{1}{p^s}\right)^{-1} $$
\end{corollary}

Observemos lo siguiente:

$$\begin{aligned}
\frac{1}{\zeta(s)} & =\prod_{p}\left(1-\frac{1}{p^s}\right) \\
& =\left(1-\frac{1}{2^s}\right)\left(1-\frac{1}{3^s}\right)\left(1-\frac{1}{5^s}\right) \ldots \\
& =1-\sum_{p} p^{-s}+\sum_{\substack{p, q\\p\neq q}} p^{-s} q^{-s}-\ldots \\
& =\sum_{n=1}^{\infty} \frac{\mu(n)}{n^s}
\end{aligned}$$

Hemos probado unicidad en series de Dirichlet, esto nos permite obtener una prueba alternativa de $\mu*1=e$. En efecto:

$$\sum_{n=1}^{\infty} \frac{e(n)}{n^s}=1=\sum_{n=1}^{\infty} \frac{\mu(n)}{n^s}\zeta(s)=\sum_{n=1}^{\infty} \frac{\mu*1(n)}{n^s}$$

Otro ejemplo de  esto es:

$$\begin{aligned}
\frac{\zeta(s-1)}{\zeta(s)} & =\prod_{p} \frac{\left(1-p^{-s}\right)}{\left(1-p^{-s+1}\right)} \\
&=\prod_{p}\left(1-\frac{1}{p^s}\right)\left(\sum_{k=0}^{\infty}\dfrac{p^k}{p^{ks}}\right)\\
& =\prod_{p}\left(1-\frac{1}{p^s}\right)\left[1+\frac{p}{p^s}+\frac{p^2}{p^{2 s}}+\ldots\right] \\
& =\prod_{p}\left(\left[1+\frac{p}{p^s}+\frac{p^2}{p^{2 s}}+\ldots\right]-\left[\frac{1}{p^s}+\frac{p}{p^{2 s}}+\frac{p^2}{p^{3 s}}+\ldots\right]\right) \\
& =\prod_{p }\left[1+\left(1-\frac{1}{p}\right)\left(\frac{p}{p^s}+\frac{p^2}{p^{2 s}}+\ldots\right)\right] \\
&=\prod_{p }\left[1+\frac{p-1}{p^s}+\frac{p^2-p}{p^{2 s}}+\ldots\right]\\
& =\sum_{n=1}^{\infty}f(n)n^{-s}.
\end{aligned}$$

Donde $f(n)$ es tal que $f(p^{k})=p^{k}-p^{k-1}$, luego, $f(n)=\varphi(n)$, ahora note que:

$$\zeta(s-1)=\sum_{n=1}^{\infty} \frac{1}{n^{s-1}}=\sum_{n=1}^{\infty} \frac{n}{n^s}$$

Así:

$$\frac{\zeta(s-1)}{\zeta(s)}=\sum_{n=1}^{\infty} \frac{\varphi(n)}{n^s}=\sum_{n=1}^{\infty} \frac{n}{n^s}\sum_{n=1}^{\infty} \frac{\mu(n)}{n^s}=\sum_{n=1}^{\infty} \frac{\mu*N(n)}{n^s}$$

Y por el teorema de unicidad, $\mu*N=\varphi$. Esto muestra el poder teórico de las series de Dirichlet, cuando tenemos propiedades adecuadas, podemos obtener mucha información de las funciones aritmética asociadas. Pero esto no es todo, mostraré una propiedad adicional, toda serie de Dirichlet se puede ver como una integral de Riemann-Stieltjes y bajo ciertas condiciones esta integral tomará la forma de una transformada de Laplace. Este hecho tomará particular importancia en la prueba del TNP.\\

\begin{lemma}[Kronecker]
Sean $f$ una función aritmética y $s\in \C$ con $\sigma>0$ tal que $F(s)$ converge, entonces:

$$\lim_{x \to \infty} \frac{1}{x^s}\sum_{n\leq x}f(n)=0$$
\end{lemma}

\begin{proof}
Sea $f$ fija pero arbitraria y $s\in \C$ con $\sigma>0$. Consideremos

$$S(x)=\sum_{n\leq x}f(n), \quad D(x)=\sum_{n\leq x}\frac{f(n)}{n^s}$$

por hipótesis $D(x)$ converge a un $D$ cuando $x\to\infty$, queremos ver que $\lim_{x \to \infty} \dfrac{S(x)}{x^s}=0$.

Dado $\epsilon>0$, existe un $x_0\geq 1$, tal que para todo $x\geq x_0$:

$$|D(x)-D|<\frac{\epsilon}{2}$$

Sea $x\geq x_0$, aplicando sumación parcial a $S(x)$ tenemos que:

\begin{align*}
S(x)&=\sum_{n\leq x}\frac{f(n)}{n^s}n^s=\int_1^xt^sd(D(t))=D(x)x^s-\int_1^xD(t)st^{s-1}dt\\
&=\int_0^xD(x)st^{s-1}dt-\int_1^xD(t)st^{s-1}dt
\end{align*}

Note que $D(t)=0$ para todo $t\in [0,1)$, luego en la segunda integral podemos cambiar el límite inferior por 0, obteniendo que:

\begin{align*}
|S(x)|&=\left|\int_0^x(D(x)-D(t))st^{s-1}dt\right|\\
&\leq \int_0^x \left|D(x)-D(t)\right|\left|s\right|t^{\sigma-1}dt.
\end{align*}

Para $x_0\leq t\leq x$ tenemos que:

$$|D(t)-D(x)|\leq |D(t)-D|+|D-D(x)|<\frac{\epsilon}{2}+\frac{\epsilon}{2}=\epsilon$$

y si $0\leq t\leq x_0$, entonces:

\begin{align*}
|D(t)-D(x)|&\leq|D(t)|+|D|+|D-D(x)|\\
&<\sum_{n\leq x_0}\left|\frac{f(n)}{n^s}\right|+|D|+\frac{\epsilon}{2}=M  
\end{align*}

donde $M$ depende de $\epsilon$, pero no de $x$. De esto se sigue que:

$$
\begin{aligned}
|S(x)| & \leq \epsilon \int_{x_0}^x|s| t^{\sigma-1} d t+M \int_0^{x_0}|s| t^{\sigma-1}dt\\
& \leq \frac{|s|}{\sigma}\left(\epsilon\left(x^{\sigma}-x_0^{\sigma}\right)+M x_0^{\sigma}\right)
\end{aligned}
$$

así:

\begin{equation}
\left|\frac{S(x)}{x^s}\right| \leq \frac{|s|}{\sigma}\left(\epsilon+\frac{M x_0^{\sigma}}{x^{\sigma}}\right)
\end{equation}

Como $M$ no depende de $x$, entonces tomando $x\to\infty$, el último término en (1.15) tiende a 0, luego para todo $\epsilon>0$:

$$\lim_{x \to \infty}\left|\frac{S(x)}{x^s}\right|\leq \frac{\epsilon|s|}{\sigma}$$

y como $\epsilon$ es arbitrario, $\displaystyle\lim_{x \to \infty} \frac{S(x)}{x^s}=0$.\\
\end{proof}

Sea $F(s)$ una serie de Dirichlet con abscisa de convergencia $\sigma_c>0$, para todo $s\in \C$ con $\sigma>\sigma_c$:

\begin{align*}
\sum_{n=1}^{\infty}\frac{f(n)}{n^s}&=\lim_{N \to \infty} \int_1^N x^{-s}d(S(x))\\
&=\lim_{N \to \infty}\frac{S(N)}{N^s}-\lim_{N \to \infty} \int_1^N S(x)d(x^{-s})\\
&=s\int_1^{\infty} \frac{S(x)}{x^{s+1}}dx
\end{align*}

el término de borde se anula por el lema de Kronecker. Así obtenemos una representación integral para cualquier serie de Dirichlet con abscisa de convergencia positiva. Aplicando esto a $\zeta(s)$  obtenemos una extensión analítica de la función zeta de Riemann al semiplano $\sigma>0$ con un polo simple en $s=1$ con residuo 1.\\

Note que tomando $x=e^t$, $dx=e^tdt$

$$\sum_{n=1}^{\infty} \frac{f(n)}{n^s}=s\int_0^{\infty}S(e^{t})e^{-st}dt$$

la serie de Dirichlet toma forma de transformada de Laplace. Aunque esto parezca una propiedad cualquiera, Korevaar y Zagier aplicaron esto para obtener una prueba corta del TNP en la que no es necesario el teorema Tauberiano de Wiener-Ikehara \cite{zagier1997newman}, estas ideas también son las que se siguen en el libro de Pongsriiam \cite{pongsriiam2023analytic}.\\

Ahora, vamos usar  todo lo anterior para obtener propiedades de $\zeta(s)$ que necesitamos para la prueba del TNP.

\section{La función zeta de Riemann}

\begin{theorem}[Propiedades]
Sea $s\in\C$ tal que $\sigma>1$, entonces:

\begin{itemize}[]
\item[i)] $\zeta(s)\neq 0$
\item[ii)] $\displaystyle\zeta^{\prime}(s)=-\sum_{n=1}^{\infty} \frac{\log n}{n^s}$
\item[iii)] $\displaystyle\frac{\zeta^{\prime}(s)}{\zeta(s)}=-\sum_{n=1}^{\infty} \frac{\Lambda(n)}{n^s}$
\end{itemize}
\end{theorem}

Las propiedades (ii) y (iii) ya las hemos probado antes, basta ver (i).\\

\begin{proof}
Como $\sigma>1$, entonces:

$$\zeta(s)=\prod_p  \frac{1}{1-p^{-s}} $$

donde el producto es absolutamente convergente, note que:

$$\frac{1}{1-p^{-s}}=\frac{p^s}{p^s-1}=1+\frac{1}{p^s-1}$$

entonces el producto no tiene factores nulos, luego no converge a 0, $\zeta(s)\neq 0$
\end{proof}

\begin{theorem}
Sea $s\in \C$ tal que $\sigma>1$, entonces:

$$\zeta(s)=\frac{s}{s-1}-s\int_1^{\infty}\frac{\{t\}}{t^{s+1}}dt$$
\end{theorem}

\begin{proof}
Tenemos que $\zeta(s)=\displaystyle s\int_1^{\infty}\frac{[t]}{t^{s+1}}dt$, como $[t]=t-\{t\}$, entonces:

\begin{align*}
\zeta(s)&=s\int_1^{\infty}\frac{t-\{t\}}{t^{s+1}}\\
&=s\int_1^{\infty}\frac{1}{t^s}dt-s\int_1^{\infty}\frac{\{t\}}{t^{s+1}}dt\\
&=\frac{s}{s-1}-s\int_1^{\infty}\frac{\{t\}}{t^{s+1}}dt
\end{align*}
\end{proof}

\begin{theorem}[Extensión analítica]
La representación integral de $\zeta(s)$ nos da una extensión analítica al semiplano $\sigma>0$ con un polo simple en $s=1$ con residuo 1.
\end{theorem}

\begin{proof}
Note que $\dfrac{s}{s-1}=1+\dfrac{1}{s-1}$ es analítica en el semiplano $\sigma>0$ excepto por un polo simple en $s=1$ con residuo 1. Basta ver que la función:

$$f(s)=\int_1^{\infty}\frac{\{t\}}{t^{s+1}}dt$$

es analítica en el semiplano $\sigma>0$. Para cada $n\in \N$, sea $f_n(s)=\displaystyle\int_1^{n}\frac{\{t\}}{t^{s+1}}dt$, tenemos que:

$$f_n(s)=\int_1^{n}\{t\}e^{\displaystyle-(s+1)\log t}dt=\int_1^{n}\sum_{n=0}^{\infty}\frac{\{t\}(-\log t)^n(s+1)^n}{n!}dt$$

donde la suma dentro de la integral es la serie de Taylor de $e^x$ que sabemos tiene radio de convergencia infinito. Queremos ver que $f_n$ es una sucesión de funciones analíticas, para lo que basta ver que podemos introducir la integral en la serie de potencias, además queremos ver que $f_n\to f$ uniformemente en cualquier subconjunto compacto de $\sigma>0$. En efecto:

$$\begin{aligned}
\sum_{m=0}^{\infty}\left|\frac{\{t\}(-\log t)^m(s+1)^m}{m!}\right| & \leq \sum_{m=0}^{\infty} \frac{(|\log t|(|s|+1))^m}{m!} \\
& =e^{\displaystyle\left| \log t\right|(|s|+1)}=t^{|s|+1}
\end{aligned}$$

entonces 

$$\int_1^n \sum_{m=0}^{\infty}\left|\frac{\{t\}(-\log t)^m(s+1)^m}{m!}\right| d t \leq \int_1^n t^{|s|+1} d t<\infty$$

luego por convergencia absoluta:

$$f_n(s)=\sum_{m=0}^{\infty} \frac{(s+1)^m}{m!} \int_1^n\{t\}(-\log t)^m d t.$$

Sea $s\in \C$ con $\sigma\geq\delta>0$. Dado $\epsilon>0$, sea $N=(\delta\epsilon)^{-\frac{1}{\delta}}$, entonces para todo $n>N$:

\begin{align*}
\left|f(s)-f_n(s)\right| &\leq \int_n^{\infty} \left|\frac{\{t\}}{t^{s+1}}\right| d t \leq \int_n^{\infty}\frac{1}{t^{\sigma+1}}\leq \frac{1}{\sigma n^\sigma}\\
&\leq \frac{1}{\delta n^\delta}\\
&<\frac{1}{\delta(\delta\epsilon)^{-\frac{1}{\delta}\delta}}=\epsilon
\end{align*}

como $N$ no depende de $s$, $f_n$ converge uniformemente a $f$ en el semiplano $\sigma\geq\delta$, en particular sobre cualquier compacto de $\sigma>0$, por el Teorema 1.26 $f(s)$ es analítica en el semiplano $\sigma>0$.

\end{proof}

\subsection{La ecuación funcional}

Esta parte es opcional, no es necesaria para  la prueba del TNP, no estoy seguro de ponerla.